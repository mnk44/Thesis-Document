\documentclass[12pt,letterpaper]{report}

\usepackage[utf8]{inputenc}
\usepackage[spanish]{babel}
\usepackage{graphicx}
\usepackage[left=3cm,right=3cm,top=2cm,bottom=2cm]{geometry}
\usepackage[breaklinks,colorlinks=true,linkcolor=black,citecolor=blue,urlcolor=black]{hyperref}

\makeindex

\title{Título tesis}

\renewcommand{\baselinestretch}{1.2}
\begin{document}
	\pagenumbering{roman}	
	\renewcommand{\listtablename}{Índice de tablas}
	\renewcommand{\tablename}{Tabla}
	\renewcommand{\bibname}{Referencias bibliográficas}	
%	esto es para que las listas anidadas salgan con numeros
	\renewcommand{\labelenumii}{\arabic{enumi}.\arabic{enumii}}
	\renewcommand{\labelenumiii}{\arabic{enumi}.\arabic{enumii}.\arabic{enumiii}}
	\renewcommand{\labelenumiv}{\arabic{enumi}.\arabic{enumii}.\arabic{enumiii}.\arabic{enumiv}}
	
	
	\pagestyle{empty}		
	\thispagestyle{empty} 
	
 	\begin{titlepage}

\centering

{\includegraphics[width=0.5\textwidth]{imagen/cujae}\par}

\vspace{3.5cm}

{\LARGE Sistema para el entrenamiento de operarios en la Industria Alimentaria Cubana\par}
\vspace{1cm}

{\bfseries\Large Trabajo de diploma para optar por el título de Ingeniería en Informática\par}
\vspace{2.5cm}

{\bfseries\Large Autora:} { \Large Mónica Montoto Montané \break \href{mailto:mmontoto@ceis.cujae.edu.cu}{\textcolor{blue}{mmontoto@ceis.cujae.edu.cu}}
\par}
{\bfseries\Large Tutora: }{\Large Dra. Raisa Socorro Llanes \break
\href{mailto:raisa@ceis.edu.cu}{\textcolor{blue}{raisa@ceis.edu.cu}} \par}
\vfill

{\Large La Habana, Cuba \par}

\large{Diciembre, 2022}

\end{titlepage}



	%\include{dec_autoria}
	%\include{opinion_tutor}
	%\chapter*{Dedicatoria}
A mis padres, por siempre estar a mi lado, por la confianza depositada, por los empujones cuando no quería seguir, por motivarme, por hacerme una persona de bien y cada día luchar por mi futuro; a ellos, se los debo todo. Una vez me dijeron que la idea de criar hijos no es para que te acompañen cuando tú estés viejo, es para asumir la responsabilidad de criar humanos funcionales, comprometidos con la naturaleza y la sociedad. Este título también es de ustedes, porque lo sufrieron tanto como yo, y se los entrego con la promesa de que seguiré intentando ser una persona de bien, y que se sientan tan orgullosos como yo estoy de ustedes.

A mi tutora Raisa, que aceptó el reto de cuidarme y apoyarme, y lo hizo de la mejor manera. Sin dudas puedo decir, que tuve la suerte de conocer a una de las mejores profesionales que trabajan en la universidad. Gracias por enseñarme, por apoyarme y tenerme paciencia. A usted no solo la quiero, yo la admiro.
	%\chapter*{Agradecimientos}
Quiero agradecer a mi familia por todo el apoyo que recibí a lo largo de mi carrera. A mis padres por siempre estar a mi lado. A mi hermano por acercarse a preguntar y darme ánimos. A mis abuelos, por consentirme en todo lo que pueden y por llamar todos los días para preguntar cómo estoy, cómo me siento, si dormí, si comí, cuánto me falta; por ser los mejores abuelos, los amo. A los primos que llamaban para preguntar cómo estoy y cómo me va. A mi tía, que me ayudó siempre que lo necesité. A Alejandro López Rodríguez por estar a mi lado, por darme ánimos y por creer en mi hasta cuando ni yo creía, y a sus padres, por todo el apoyo, la preocupación y la ayuda que me han brindado. Gracias a todos, por ser una familia tan unida y, aunque no somos perfectos, gracias por nunca faltar.

También quiero agradecer a mis amigos, sin ellos soy nada.

Agradecer a Aramays Aimet Morales Durán, que cuando vea su segundo nombre se va a enojar. Ella, más que una amiga, es mi hermana. No solo me consuela y me aconseja, ella fue mi guía todos estos años de universidad. Era la que me motivaba, la que me decía cuando había algún pendiente que olvidé, la que hacía que me ``pusiera para las cosas", fue mi estrella en la facultad.

Agradecer también a Oscar García Páez que, aunque no siga en la carrera es una de las mejores cosas que me pasó. Es mi fiel amigo, el más sincero, el que no teme decirme las verdades a la cara, aunque sepa que no me gusta lo que voy a escuchar. Mi historiador, gracias por ser mi hermano mayor.

A Pavel Pérez González, nuestro ``jefazo". Por ser un gran amigo y compañero, por esos consejos tan únicos e interesantes, por siempre ser atento. Es la persona a la que acudir cuando tienes una duda, porque lo que no sabe, lo aprende con tal de ayudar. Gracias.

A mis amigos y compañeros de aula: Adrián, Abel, Néstor, Daniela, Roly, Darío, Thalía, Amanda y algún otro que se me olvide mencionar. Gracias por la ayuda, las risas y los trabajos en equipo, fue la mejor parte de la universidad.

Gracias a las chicas de la tabla y a mis compañeros de los trece: Claudia, Camila (La flaca), Legna, Yoan, Eduardo (El gato), Víctor, Thalía (La loca), Samuel y María. Junto a ellos defendí el amarillo de mi facultad, participando en deportes que nunca creí jugar. Gracias por enseñarme y cuidarme las lesiones.

 A las chicas del olimpo: Diansy, Thalía, Laura y Arianna, y a mis compañeros de la FEU: Mariam y Lázaro Michel (Tachiri). Cada uno se ocupó de mí, me ayudaron, compartieron conmigo eventos y marchas, y me impulsaron a ser mejor persona. Gracias.
 
Gracias a los amigos que conocí en la pandemia: Mariam Yilian, Anabel Achkienasi, Ronal, Daniela (Mi Stich), Luis Osvel y Laura. A mi compañero de apogones Yamir. En general, gracias a todos los amigos que hice en la universidad, cada uno formó parte de este proceso y son un pedazo de esta etapa tan bonita.

Quiero agradecer a mis profesores, por la formación que me dieron. En especial a Wenny por enseñarme que siempre se puede más, a Rosete por su arte de hacernos amar la carrera, a Mayi por ser como una madre para todos, a Sonia y a Nayma.

Por último, un agradecimiento muy especial a mi tutora, por tenerme tanta paciencia. Gracias por entenderme y apoyarme, aún cuando llevase días desaparecida. Gracias por siempre decirme que puedo y hacerme creer que el mundo es mío.

A todos los que formaron parte de este proceso, de todo corazón: \textbf{Gracias}.
	\include{resumen}
	\section*{Abstract}
Job training is a method applied by companies so that their workers acquire new professional knowledge. There are currently a set of systems that allow this training to be carried out automatically, without the mandatory presence of a superior. In this way, many valuable resources are saved, such as time and human capital.
In industries where work is uninterrupted, having these systems represents a notable advantage in production, hence the need to incorporate them as permanent tools.
When it comes to evaluating the results obtained, in general, automated training systems have an automatic evaluation method incorporated. One of the most used methods is the use of expert systems.

The food industry of a country plays an extremely important role within the economy. In Cuba, this industry is characterized by its constant work and its changing personnel. Due to these characteristics, it is almost impossible to achieve face-to-face training for employees, which means that not all of them have the same level of knowledge and therefore, not all of them can respond to certain situations.

The Expert System for the Control of Chemical Processes (SECPROIT) is a software created for the training of operators before the productive processes of the Cuban Food Industry. In it, a set of trainings are applied where the response capacity of the workers in critical situations and their knowledge about the processes are evaluated. The results are evaluated through the use of expert systems. It currently contains a group of restrictions that prevent its use in industries.

This research aims to solve the existing limitations in the SECPROIT system.

\vfill

\begin{description}
	\item[Keywords:]{job training, automated training systems, training systems, expert systems, Expert System for the Control of Chemical Processes (SECPROIT), Knowledge Base Generator System (SGBC)}
\end{description}

	
	\pagestyle{plain}	
	
	\tableofcontents	
	\pagebreak	
	
	\listoftables
	\pagebreak
	
	\listoffigures	
 	\clearpage 

	\pagenumbering{arabic}
	
	\phantomsection
	\addcontentsline{toc}{chapter}{Introducción}
	\pagestyle{fancy} 
	
	\chapter*{Introducción}
En la actualidad, hablar de tecnología es pensar en herramientas que hace diez años parecían futuristas y muy difíciles de conseguir, sin embargo, hoy es posible lograr cambios significativos en empresas y hogares, gracias a los avances tecnológicos \cite{Casamitjana2021}. Cada vez es más común escuchar sobre la revolución tecnológica, que crece de manera exponencial y pretende transformar por completo el sector industrial. Estos desarrollos van desde la mecanización de tareas hasta los procesos industriales autónomos, en los que no se necesita la intervención humana para realizar un trabajo \cite{Rani2022}.

Para mediados del año 2019, solo un 31\% de las industrias en América Latina y el Caribe poseían procesos automatizados o avances digitales. Esta cifra cambió bruscamente al año siguiente, debido a que la pandemia de la COVID-19 dejó una huella devastadora en la economía. Las empresas que presentaban un avance digital pudieron resistir mejor a la crisis, en términos de impacto sobre las ventas, beneficios y trabajadores despedidos. A partir de ese momento, muchas instituciones tomaron conciencia y comenzaron a incluir nuevos avances tecnológicos en sus procesos productivos. Hoy en día, el 86\% de las industrias latinoamericanas cuentan con un desarrollo industrial automatizado y se espera que este cambio se mantenga o aumente en el futuro \cite{Yong2021}.

Entre las industrias más importantes de un país se encuentra la industria alimentaria, que comprende el sector de la economía que se ocupa de todos aquellos procesos relacionados con la alimentación de las personas. Está encaminada a satisfacer la necesidad más básica de la población, sin la cual no podríamos sobrevivir: la nutrición. Mundialmente, posee como características la gran variedad de materias primas que emplea, el número elevado de procesos que maneja y su continuo crecimiento, relacionado al ritmo con el que crece la población \cite{ValRoman2016}. 

Cuando se detiene de forma no programada un proceso productivo en una empresa, por lo general trae consigo numerosas pérdidas. En el caso de las industrias alimentarias, estas afectaciones pueden verse reflejadas en grandes daños económicos ya que, en la mayoría de los casos, este tipo de industrias trabajan con materias que no pueden ser reutilizadas; pero no son las únicas afectaciones que se pueden encontrar. En sus procesos productivos, se elaboran distintos tipos de alimentos a partir de un producto determinado, donde cada producto pertenece a una escala de riesgo (A, B o C). Esta escala indica la probabilidad que tiene el proceso de causar daños en la salud (A para alta, B para media y C para baja). Es decir, si se detiene un proceso productivo en la industria alimentaria no solo traería consigo afectaciones en la economía, sino que también puede verse afectada la salud de los trabajadores o incluso de la población \cite{Salas2018}.

La Industria Alimentaria Cubana se caracteriza, además, por poseer jornadas laborales ininterrumpidas y un personal que cambia frecuentemente. Esta situación dificulta la capacitación de sus trabajadores, por lo que no todos los que laboran poseen el mismo nivel de conocimiento. Como consecuencia, al ocurrir un error en un proceso productivo, no siempre se encuentra el experto capaz de solucionar la falla, dejando como única alternativa: detener el proceso \cite{Lemus2018}.

Partiendo de esa dificultad, el Instituto de Investigación de la Industria Alimentaria (IIIA), junto con las facultades de Ingeniería Química e Ingeniería Informática de la Universidad Tecnológica de La Habana (CUJAE), desarrolló dos sistemas de software: un Sistema Generador de Bases de Conocimiento (SGBC) y  un Sistema Experto para el Control de Procesos Químicos (SECPROIT). El primero genera bases de conocimiento que contienen la información referente a los procesos productivos que pueden ocurrir en una fábrica, estructurada en: variables, causas por las que puedan estar en peligro y recomendaciones (pasos a seguir en caso de accidentes)  \cite{Riveron2017}. El segundo, se utiliza para producir entrenamientos y capacitar a los operarios de la fábrica \cite{ElenaAcostaGil2018}. Con estos sistemas, los trabajadores se entrenarían sobre los procesos productivos que maniobran y se reducirían los riesgos de errores por desconocimiento. Lamentablemente, el sistema SECPROIT posee un grupo restricciones que impidieron que fuese implementado en las industrias.

Por lo tanto, la situación problemática presente en la Industria Alimentaria Cubana sigue siendo la misma, puesto que no se ha podido implementar un sistema de entrenamiento estable que solucione el problema. Partiendo de este principio, una solución viable sería desarrollar una actualización funcional del SECPROIT, que pueda resolver el problema de investigación inicial: 
¿Cómo lograr una capacitación total de los operarios, ante la toma de decisiones en una situación crítica?

En función de esa problemática, el objetivo general de esta investigación es: 
\textsl{Rectificar las limitaciones existentes en el sistema SECPROIT para capacitar a los operarios ante los procesos productivos de la fábrica.}

Analizando las restricciones que posee el SECPROIT, se perciben un número de problemas que son necesarios enmendar. Estas modificaciones deben realizarse tanto en la lógica, como en la estructura, en la base de datos y en las interfaces de usuario. Es decir, se necesitan transformar todas las capas del sistema.  Según \cite{Plecka2013}, cambiar un código ajeno, con el fin de añadirle una nueva configuración, adaptarlo a una nueva disposición y modificar su base de información, resulta más complejo que elaborar uno de cero. Por lo tanto, con el fin de rectificar las limitaciones actuales y generar una actualización, resulta más factible comenzar un nuevo software, siguiendo el modelo existente.

Tomando en cuenta esta premisa, se definen los siguientes objetivos específicos y sus tareas correspondientes:
\begin{enumerate}
\item Asimilar los fundamentos teóricos y analizar los nuevos aspectos que puedan ser incorporados
\begin{itemize}
\item Extraer los requisitos funcionales y no funcionales que debe cumplir el nuevo sistema
\item Investigar los diferentes tipos de preguntas existentes en un
sistema de entrenamiento y seleccionar los que mejor se ajusten al proceso evaluativo del SECPROIT
\end{itemize}

\item Desarrollar el nuevo sistema
\begin{itemize}
\item Diseñar diagramas auxiliares y modelados UML
\item Implementar la base de datos
\item Desarrollar funcionalidades que permitan la administración del sistema
\item Incorporar la configuración de un entrenamiento asociado a un proceso
\item Implementar la generación de preguntas para cada fase del entrenamiento
\item Desarrollar un proceso de evaluación parcial para cada etapa y la evaluación integral del operario en un proceso
\item Diseñar e incluir una nueva interfaz gráfica
\end{itemize}

\item Validar el nuevo sistema
\begin{itemize}
\item Diseñar un conjunto de pruebas que verifiquen el correcto funcionamiento del sistema
\item Ejecutar las pruebas diseñadas
\item Documentar los resultados obtenidos
\end{itemize}
\end{enumerate}

El objeto de estudio de esta investigación son los sistemas de entrenamiento automatizados, los sistemas expertos y los sistemas de información basados en reglas de producción. De ellos, se centra la atención en el campo de acción que comprende el Sistema Experto para el Control de Procesos Químicos (SECPROIT).

Este proyecto presenta un aporte tanto en el marco teórico como en el práctico. En el marco teórico, se construye un sistema de entrenamiento capaz de ejercitar sobre cualquier información que se brinde en sus bases de conocimiento. En la práctica, se desarrolla un sistema de software para la capacitación de los operarios de la Industria Alimentaria Cubana.

En cuanto a la estructura de este trabajo, está dividido en tres capítulos. En el capítulo 1 se encuentran los conceptos fundamentales y los antecedentes de esta investigación. El capítulo 2 describe el diseño e implementación del nuevo sistema, utilizando diferentes artefactos UML, describiendo los requisitos funcionales y no funcionales y mostrando la vista de la arquitectura y los patrones utilizados. Por último, en el capítulo 3 se presenta la documentación de las pruebas que se le realizaron al software.	
	

	
\end{document}