\documentclass[12pt,letterpaper]{report}

\usepackage[utf8]{inputenc}
\usepackage[spanish]{babel}
\usepackage{graphicx}
\usepackage{float}
\usepackage{array}
\usepackage[left=3cm,right=3cm,top=2cm,bottom=2cm]{geometry}
\usepackage[breaklinks,colorlinks=true,linkcolor=black,citecolor=blue,urlcolor=black]{hyperref}

\makeindex

\title{Tesis}

\renewcommand{\baselinestretch}{1.2}
\begin{document}
	\pagenumbering{roman}	
	\renewcommand{\listtablename}{Índice de tablas}
	\renewcommand{\tablename}{Tabla}
	\renewcommand{\bibname}{Referencias bibliográficas}	
%	esto es para que las listas anidadas salgan con numeros
	\renewcommand{\labelenumii}{\arabic{enumi}.\arabic{enumii}}
	\renewcommand{\labelenumiii}{\arabic{enumi}.\arabic{enumii}.\arabic{enumiii}}
	\renewcommand{\labelenumiv}{\arabic{enumi}.\arabic{enumii}.\arabic{enumiii}.\arabic{enumiv}}
	
	
	\pagestyle{empty}		
	\thispagestyle{empty} 
	
 	\begin{titlepage}

\centering

{\includegraphics[width=0.5\textwidth]{imagen/cujae}\par}

\vspace{3.5cm}

{\LARGE Sistema para el entrenamiento de operarios en la Industria Alimentaria Cubana\par}
\vspace{1cm}

{\bfseries\Large Trabajo de diploma para optar por el título de Ingeniería en Informática\par}
\vspace{2.5cm}

{\bfseries\Large Autora:} { \Large Mónica Montoto Montané \break \href{mailto:mmontoto@ceis.cujae.edu.cu}{\textcolor{blue}{mmontoto@ceis.cujae.edu.cu}}
\par}
{\bfseries\Large Tutora: }{\Large Dra. Raisa Socorro Llanes \break
\href{mailto:raisa@ceis.edu.cu}{\textcolor{blue}{raisa@ceis.edu.cu}} \par}
\vfill

{\Large La Habana, Cuba \par}

\large{Diciembre, 2022}

\end{titlepage}



	%\include{dec_autoria}
	%\include{opinion_tutor}
	%\chapter*{Dedicatoria}
A mis padres, por siempre estar a mi lado, por la confianza depositada, por los empujones cuando no quería seguir, por motivarme, por hacerme una persona de bien y cada día luchar por mi futuro; a ellos, se los debo todo. Una vez me dijeron que la idea de criar hijos no es para que te acompañen cuando tú estés viejo, es para asumir la responsabilidad de criar humanos funcionales, comprometidos con la naturaleza y la sociedad. Este título también es de ustedes, porque lo sufrieron tanto como yo, y se los entrego con la promesa de que seguiré intentando ser una persona de bien, y que se sientan tan orgullosos como yo estoy de ustedes.

A mi tutora Raisa, que aceptó el reto de cuidarme y apoyarme, y lo hizo de la mejor manera. Sin dudas puedo decir, que tuve la suerte de conocer a una de las mejores profesionales que trabajan en la universidad. Gracias por enseñarme, por apoyarme y tenerme paciencia. A usted no solo la quiero, yo la admiro.
	%\chapter*{Agradecimientos}
Quiero agradecer a mi familia por todo el apoyo que recibí a lo largo de mi carrera. A mis padres por siempre estar a mi lado. A mi hermano por acercarse a preguntar y darme ánimos. A mis abuelos, por consentirme en todo lo que pueden y por llamar todos los días para preguntar cómo estoy, cómo me siento, si dormí, si comí, cuánto me falta; por ser los mejores abuelos, los amo. A los primos que llamaban para preguntar cómo estoy y cómo me va. A mi tía, que me ayudó siempre que lo necesité. A Alejandro López Rodríguez por estar a mi lado, por darme ánimos y por creer en mi hasta cuando ni yo creía, y a sus padres, por todo el apoyo, la preocupación y la ayuda que me han brindado. Gracias a todos, por ser una familia tan unida y, aunque no somos perfectos, gracias por nunca faltar.

También quiero agradecer a mis amigos, sin ellos soy nada.

Agradecer a Aramays Aimet Morales Durán, que cuando vea su segundo nombre se va a enojar. Ella, más que una amiga, es mi hermana. No solo me consuela y me aconseja, ella fue mi guía todos estos años de universidad. Era la que me motivaba, la que me decía cuando había algún pendiente que olvidé, la que hacía que me ``pusiera para las cosas", fue mi estrella en la facultad.

Agradecer también a Oscar García Páez que, aunque no siga en la carrera es una de las mejores cosas que me pasó. Es mi fiel amigo, el más sincero, el que no teme decirme las verdades a la cara, aunque sepa que no me gusta lo que voy a escuchar. Mi historiador, gracias por ser mi hermano mayor.

A Pavel Pérez González, nuestro ``jefazo". Por ser un gran amigo y compañero, por esos consejos tan únicos e interesantes, por siempre ser atento. Es la persona a la que acudir cuando tienes una duda, porque lo que no sabe, lo aprende con tal de ayudar. Gracias.

A mis amigos y compañeros de aula: Adrián, Abel, Néstor, Daniela, Roly, Darío, Thalía, Amanda y algún otro que se me olvide mencionar. Gracias por la ayuda, las risas y los trabajos en equipo, fue la mejor parte de la universidad.

Gracias a las chicas de la tabla y a mis compañeros de los trece: Claudia, Camila (La flaca), Legna, Yoan, Eduardo (El gato), Víctor, Thalía (La loca), Samuel y María. Junto a ellos defendí el amarillo de mi facultad, participando en deportes que nunca creí jugar. Gracias por enseñarme y cuidarme las lesiones.

 A las chicas del olimpo: Diansy, Thalía, Laura y Arianna, y a mis compañeros de la FEU: Mariam y Lázaro Michel (Tachiri). Cada uno se ocupó de mí, me ayudaron, compartieron conmigo eventos y marchas, y me impulsaron a ser mejor persona. Gracias.
 
Gracias a los amigos que conocí en la pandemia: Mariam Yilian, Anabel Achkienasi, Ronal, Daniela (Mi Stich), Luis Osvel y Laura. A mi compañero de apogones Yamir. En general, gracias a todos los amigos que hice en la universidad, cada uno formó parte de este proceso y son un pedazo de esta etapa tan bonita.

Quiero agradecer a mis profesores, por la formación que me dieron. En especial a Wenny por enseñarme que siempre se puede más, a Rosete por su arte de hacernos amar la carrera, a Mayi por ser como una madre para todos, a Sonia y a Nayma.

Por último, un agradecimiento muy especial a mi tutora, por tenerme tanta paciencia. Gracias por entenderme y apoyarme, aún cuando llevase días desaparecida. Gracias por siempre decirme que puedo y hacerme creer que el mundo es mío.

A todos los que formaron parte de este proceso, de todo corazón: \textbf{Gracias}.
	
	\include{resumen}
	\section*{Abstract}
Job training is a method applied by companies so that their workers acquire new professional knowledge. There are currently a set of systems that allow this training to be carried out automatically, without the mandatory presence of a superior. In this way, many valuable resources are saved, such as time and human capital.
In industries where work is uninterrupted, having these systems represents a notable advantage in production, hence the need to incorporate them as permanent tools.
When it comes to evaluating the results obtained, in general, automated training systems have an automatic evaluation method incorporated. One of the most used methods is the use of expert systems.

The food industry of a country plays an extremely important role within the economy. In Cuba, this industry is characterized by its constant work and its changing personnel. Due to these characteristics, it is almost impossible to achieve face-to-face training for employees, which means that not all of them have the same level of knowledge and therefore, not all of them can respond to certain situations.

The Expert System for the Control of Chemical Processes (SECPROIT) is a software created for the training of operators before the productive processes of the Cuban Food Industry. In it, a set of trainings are applied where the response capacity of the workers in critical situations and their knowledge about the processes are evaluated. The results are evaluated through the use of expert systems. It currently contains a group of restrictions that prevent its use in industries.

This research aims to solve the existing limitations in the SECPROIT system.

\vfill

\begin{description}
	\item[Keywords:]{job training, automated training systems, training systems, expert systems, Expert System for the Control of Chemical Processes (SECPROIT), Knowledge Base Generator System (SGBC)}
\end{description}

	
	\pagestyle{plain}	
	
	\tableofcontents	
	\pagebreak	
	
	\listoftables
	\pagebreak
	
	\listoffigures	
 	\clearpage 

	\pagenumbering{arabic}
	
	\phantomsection
	\addcontentsline{toc}{chapter}{Introducción}
	\pagestyle{fancy} 
	
	\chapter*{Introducción}
En la actualidad, hablar de tecnología es pensar en herramientas que hace diez años parecían futuristas y muy difíciles de conseguir, sin embargo, hoy es posible lograr cambios significativos en empresas y hogares, gracias a los avances tecnológicos \cite{Casamitjana2021}. Cada vez es más común escuchar sobre la revolución tecnológica, que crece de manera exponencial y pretende transformar por completo el sector industrial. Estos desarrollos van desde la mecanización de tareas hasta los procesos industriales autónomos, en los que no se necesita la intervención humana para realizar un trabajo \cite{Rani2022}.

Para mediados del año 2019, solo un 31\% de las industrias en América Latina y el Caribe poseían procesos automatizados o avances digitales. Esta cifra cambió bruscamente al año siguiente, debido a que la pandemia de la COVID-19 dejó una huella devastadora en la economía. Las empresas que presentaban un avance digital pudieron resistir mejor a la crisis, en términos de impacto sobre las ventas, beneficios y trabajadores despedidos. A partir de ese momento, muchas instituciones tomaron conciencia y comenzaron a incluir nuevos avances tecnológicos en sus procesos productivos. Hoy en día, el 86\% de las industrias latinoamericanas cuentan con un desarrollo industrial automatizado y se espera que este cambio se mantenga o aumente en el futuro \cite{Yong2021}.

Entre las industrias más importantes de un país se encuentra la industria alimentaria, que comprende el sector de la economía que se ocupa de todos aquellos procesos relacionados con la alimentación de las personas. Está encaminada a satisfacer la necesidad más básica de la población, sin la cual no podríamos sobrevivir: la nutrición. Mundialmente, posee como características la gran variedad de materias primas que emplea, el número elevado de procesos que maneja y su continuo crecimiento, relacionado al ritmo con el que crece la población \cite{ValRoman2016}. 

Cuando se detiene de forma no programada un proceso productivo en una empresa, por lo general trae consigo numerosas pérdidas. En el caso de las industrias alimentarias, estas afectaciones pueden verse reflejadas en grandes daños económicos ya que, en la mayoría de los casos, este tipo de industrias trabajan con materias que no pueden ser reutilizadas; pero no son las únicas afectaciones que se pueden encontrar. En sus procesos productivos, se elaboran distintos tipos de alimentos a partir de un producto determinado, donde cada producto pertenece a una escala de riesgo (A, B o C). Esta escala indica la probabilidad que tiene el proceso de causar daños en la salud (A para alta, B para media y C para baja). Es decir, si se detiene un proceso productivo en la industria alimentaria no solo traería consigo afectaciones en la economía, sino que también puede verse afectada la salud de los trabajadores o incluso de la población \cite{Salas2018}.

La Industria Alimentaria Cubana se caracteriza, además, por poseer jornadas laborales ininterrumpidas y un personal que cambia frecuentemente. Esta situación dificulta la capacitación de sus trabajadores, por lo que no todos los que laboran poseen el mismo nivel de conocimiento. Como consecuencia, al ocurrir un error en un proceso productivo, no siempre se encuentra el experto capaz de solucionar la falla, dejando como única alternativa: detener el proceso \cite{Lemus2018}.

Partiendo de esa dificultad, el Instituto de Investigación de la Industria Alimentaria (IIIA), junto con las facultades de Ingeniería Química e Ingeniería Informática de la Universidad Tecnológica de La Habana (CUJAE), desarrolló dos sistemas de software: un Sistema Generador de Bases de Conocimiento (SGBC) y  un Sistema Experto para el Control de Procesos Químicos (SECPROIT). El primero genera bases de conocimiento que contienen la información referente a los procesos productivos que pueden ocurrir en una fábrica, estructurada en: variables, causas por las que puedan estar en peligro y recomendaciones (pasos a seguir en caso de accidentes)  \cite{Riveron2017}. El segundo, se utiliza para producir entrenamientos y capacitar a los operarios de la fábrica \cite{ElenaAcostaGil2018}. Con estos sistemas, los trabajadores se entrenarían sobre los procesos productivos que maniobran y se reducirían los riesgos de errores por desconocimiento. Lamentablemente, el sistema SECPROIT posee un grupo restricciones que impidieron que fuese implementado en las industrias.

Por lo tanto, la situación problemática presente en la Industria Alimentaria Cubana sigue siendo la misma, puesto que no se ha podido implementar un sistema de entrenamiento estable que solucione el problema. Partiendo de este principio, una solución viable sería desarrollar una actualización funcional del SECPROIT, que pueda resolver el problema de investigación inicial: 
¿Cómo lograr una capacitación total de los operarios, ante la toma de decisiones en una situación crítica?

En función de esa problemática, el objetivo general de esta investigación es: 
\textsl{Rectificar las limitaciones existentes en el sistema SECPROIT para capacitar a los operarios ante los procesos productivos de la fábrica.}

Analizando las restricciones que posee el SECPROIT, se perciben un número de problemas que son necesarios enmendar. Estas modificaciones deben realizarse tanto en la lógica, como en la estructura, en la base de datos y en las interfaces de usuario. Es decir, se necesitan transformar todas las capas del sistema.  Según \cite{Plecka2013}, cambiar un código ajeno, con el fin de añadirle una nueva configuración, adaptarlo a una nueva disposición y modificar su base de información, resulta más complejo que elaborar uno de cero. Por lo tanto, con el fin de rectificar las limitaciones actuales y generar una actualización, resulta más factible comenzar un nuevo software, siguiendo el modelo existente.

Tomando en cuenta esta premisa, se definen los siguientes objetivos específicos y sus tareas correspondientes:
\begin{enumerate}
\item Asimilar los fundamentos teóricos y analizar los nuevos aspectos que puedan ser incorporados
\begin{itemize}
\item Extraer los requisitos funcionales y no funcionales que debe cumplir el nuevo sistema
\item Investigar los diferentes tipos de preguntas existentes en un
sistema de entrenamiento y seleccionar los que mejor se ajusten al proceso evaluativo del SECPROIT
\end{itemize}

\item Desarrollar el nuevo sistema
\begin{itemize}
\item Diseñar diagramas auxiliares y modelados UML
\item Implementar la base de datos
\item Desarrollar funcionalidades que permitan la administración del sistema
\item Incorporar la configuración de un entrenamiento asociado a un proceso
\item Implementar la generación de preguntas para cada fase del entrenamiento
\item Desarrollar un proceso de evaluación parcial para cada etapa y la evaluación integral del operario en un proceso
\item Diseñar e incluir una nueva interfaz gráfica
\end{itemize}

\item Validar el nuevo sistema
\begin{itemize}
\item Diseñar un conjunto de pruebas que verifiquen el correcto funcionamiento del sistema
\item Ejecutar las pruebas diseñadas
\item Documentar los resultados obtenidos
\end{itemize}
\end{enumerate}

El objeto de estudio de esta investigación son los sistemas de entrenamiento automatizados, los sistemas expertos y los sistemas de información basados en reglas de producción. De ellos, se centra la atención en el campo de acción que comprende el Sistema Experto para el Control de Procesos Químicos (SECPROIT).

Este proyecto presenta un aporte tanto en el marco teórico como en el práctico. En el marco teórico, se construye un sistema de entrenamiento capaz de ejercitar sobre cualquier información que se brinde en sus bases de conocimiento. En la práctica, se desarrolla un sistema de software para la capacitación de los operarios de la Industria Alimentaria Cubana.

En cuanto a la estructura de este trabajo, está dividido en tres capítulos. En el capítulo 1 se encuentran los conceptos fundamentales y los antecedentes de esta investigación. El capítulo 2 describe el diseño e implementación del nuevo sistema, utilizando diferentes artefactos UML, describiendo los requisitos funcionales y no funcionales y mostrando la vista de la arquitectura y los patrones utilizados. Por último, en el capítulo 3 se presenta la documentación de las pruebas que se le realizaron al software.	
	
	\renewcommand{\chaptermark}[1]{\markboth{\chaptername \, \thechapter. #1}{}}

	\chapter{Fundamentos teóricos}
El presente capítulo abarca los principales temas que se abordan a lo largo de la investigación. Se detalla qué es un sistema de capacitación automatizado, su importancia, sus tipos de preguntas y los modelos de calificación que siguen. Además, se plantea el concepto de sistemas expertos, junto a sus características más notables y su importancia. También estarán explicadas algunas de las funcionalidades del Sistema de Entrenamiento SECPROIT, sus componentes y rendimiento.
Al final del capítulo se observan conclusiones parciales a modo de resumen del mismo.

%%%%%%%%%%%%%%%%%%%%%%%%%%%%%%%%
\section{Sistemas de capacitación laboral}
La capacitación laboral es un método aplicado por las empresas para que su personal adquiera nuevos conocimientos profesionales. Por lo general, se produce ante un ascenso o incorporación, aunque no son los únicos motivos. Busca perfeccionar al colaborador en su puesto laboral, en función de las necesidades de su empresa. Es un proceso estructurado con metas bien definidas. Surge en el mundo como respuesta a la necesidad de mejorar permanentemente la calidad y formación de recursos humanos. Lo ideal es que se desarrolle de forma continua, ya que la constante formación del personal deriva en resultados positivos tanto para el grupo de trabajo como para la organización en la que se realiza \cite{Denby2010}.

\subsection{Características}
Un sistema de capacitación puede ofrecer diferentes aplicaciones en función del modelo de negocio que utilice. Su versatilidad permite adaptarse a las necesidades particulares de cada sector. Sin embargo, según \cite{GarciaPaez2022}, la mayoría de los sistemas contienen las siguientes características:

\begin{itemize}
\item Son capaces de gestionar los distintos cursos impartidos, la asistencia y la inversión en formación de la empresa.
\item Asignan a los empleados que deberán asistir y los profesionales responsables de analizar sus resultados.
\item Detectan las carencias formativas del personal antes de que influyan en el desarrollo del trabajo.
\item Clasifican las distintas actividades formativas en base a su categoría y catálogo.
\item Registran y consultan el progreso del aprendizaje de los empleados en tiempo real.
\end{itemize}

\subsection{Importancia de una buena capacitación}
La capacitación laboral juega un papel primordial para el logro de tareas y proyectos, dado que es el proceso mediante el cual los trabajadores adquieren conocimientos, herramientas, habilidades y actitudes para interactuar de forma correcta y segura en el entorno laboral. Entre los principales beneficios que aporta, según \cite{SistemasCapAsistidos}, se destacan:

\begin{itemize}
\item Calidad y mejora en el resultado de las tareas.
\item Reducción en tiempos y supervisión.
\item Solución de problemas con diferentes visiones.
\item Sensibilización ante nuevos retos.
\item Desarrollo ético y motivación del personal.
\item Seguridad y autoestima en los trabajadores.
\item Mayor especialización.
\end{itemize}

\subsection{Proceso de evaluación}
La evaluación de una capacitación no puede depender de un solo instrumento o técnica, ya que de esa forma solo se mide un tipo de aprendizaje. Los criterios para calificar que se designen, serán los porcentajes de valor que se establezcan a cada resultado de las actividades realizadas, y a su resultado final. Se debe tomar en cuenta tanto la exactitud de la respuesta, como el proceso que se siguió para llegar a la misma, así como la cantidad de intentos necesarios utilizados para hallar la solución \cite{CapTrabajadores}.

Una evaluación posee dos objetivos principales: analizar en qué medida se han cumplido los objetivos y proporcionar la reflexión de los que realizaron el entrenamiento en torno a su propio proceso de aprendizaje (metacognición). Analizar el cumplimiento de los objetivos permite detectar posibles fallas en el proceso y poder superarlas \cite{EtapasEntrenamiento}.

A modo de resumen, para obtener una correcta evaluación se deben tener en cuenta tantas herramientas como parámetros influyan.

\subsection{Fases en un proceso de evaluación de conocimiento}
Un proceso de evaluación de conocimiento, debe estar integrado por cinco etapas (Figura \ref{fig:etapasEval}), asegura \cite{EtapasEntrenamiento}. Cada una de ellas, va a marcar un conjunto de acciones, que al final se interpretarán como un buen entrenamiento:

\begin{enumerate}
\item Recogida de datos: es la recopilación sistemática de toda la información a lo largo del proceso completo de enseñanza-aprendizaje. La información recogida debe tener concordancia con los objetivos trazados, ser suficiente, representativa, relevante y ponderada, en función del peso otorgado a cada uno de los objetivos. En los sistemas en línea estas posibilidades de registrar evidencias son inmensas.
\item Puntuación de las pruebas: se realiza una vez medidos, de manera cuantitativa o cualitativa, los distintos bloques de información, con las ponderaciones, criterios e indicadores que se hayan establecido.
\item Juicio de valor: puede hacerse limitándose a criterios de grupo (evaluación normativa), refiriéndose a criterios de superación de objetivos y contenidos (evaluación de criterio), o teniendo en cuenta la personalidad, posibilidades y limitaciones del propio sujeto del aprendizaje (evaluación personalizada).
\item Toma de decisiones: habitualmente denominada calificación, se basa en la decisión a partir del resultado. Trae consigo una serie de consecuencias personales, administrativas, económicas y laborales. La acción resultante influye directamente en el adiestrado.
\item Información a los interesados: es la etapa final, que ha de llegar a diferentes destinatarios, aunque principalmente y de forma adecuada, a los capacitados. Es la confirmación de que concluye el entrenamiento.
\end{enumerate}

\begin{figure}[h]
\centering
 \includegraphics[width=0.5\linewidth]{imagen/FasesEva.jpg}
 \caption{Fases de un proceso de evaluación de conocimiento.}
 \label{fig:etapasEval} 
\end{figure}

\subsection{Proceso de validación de las respuestas}
Una vez terminada la capacitación se deben comprobar cuáles de los resultados obtenidos son correctos y cuáles no. Para ello se deben comparar las respuestas del evaluado con una fuente de confianza, que contenga la información verídica de lo que se está tratando. Estas fuentes de confianza se conocen por el nombre de: bases de conocimiento \cite{SEdiagramas}.

A partir de estas bases se verifica si los datos en las respuestas del evaluado coinciden con la información real contenida. Este proceso puede realizarse tanto de manera manual, semiautomática o automática.

%%%%%%%%%%%%%%%%%%%%%%%%%%%%%%%
\section{Sistemas de capacitación automatizados}
Basándose en el concepto de capacitación, un sistema de capacitación automatizado es un método de enseñanza alternativo creado para el adiestramiento de los trabajadores. Es un software que, principalmente, permite el aprendizaje de los usuarios sin necesidad de una supervisión constante. Por lo general, resulta más efectivo que las prácticas de enseñanza presencial, debido a que el estudiante trabaja solo y puede determinar su propia velocidad de aprendizaje, usando una amplia variedad de herramientas y métodos para la transferencia del conocimiento \cite{CapacitacionAuto}.

A modo de resumen, es un software que brinda una solución de recursos humanos, ayuda en la formación de los trabajadores y aumenta la productividad empresarial.

\subsection{Ventajas}
Un sistema de entrenamiento asistido por computadora (sistema de capacitación automatizado), permite ofrecer el mismo nivel de adiestramiento para cada usuario del sistema, en cuanto a rigor y evaluación. Uno de los problemas principales de la capacitación de los empleados de manera presencial es que las sesiones son frecuentemente inconsistentes y las diferencias en el nivel de habilidad del formador pueden tener un impacto significativo en el éxito del empleado. Al contar con un sistema automatizado, solo se necesita una base de conocimiento para garantizar el mismo nivel de entrenamiento para todos los capacitados \cite{SistemaCapAut}.

\subsection{Desventajas}

\subsection{Tipos de preguntas más utilizadas}
A medida que avanza el tiempo, se generan nuevos métodos de estudio, y con estos, nuevas formas de preguntar y calificar. Sin embargo, a la hora de diseñar un sistema automatizado, no es menos cierto que existen algunas variantes más sencillas y, por ende, más utilizadas. Según \cite{Catadores}, los tipos de preguntas que mayormente se emplean en un sistema de este tipo son:

\begin{itemize}
\item Verdadero o falso: contienen una declaración que se debe indicar si es verdadera o no. Permiten responder en poco tiempo, son fáciles, rápidos de calificar y se corrigen de forma automática.
\item Opción múltiple: se componen de una pregunta (raíz) con múltiples respuestas posibles. Pueden incluir múltiples opciones válidas, en cuyo caso, podrían darse por superada al marcar cualquiera de ellas o cuando se marquen todas. Se caracterizan por ser fáciles y rápidas de calificar, por corregirse automáticamente y utilizarse para evaluar los conocimientos en una amplia gama de contenidos.
\item Emparejar, relacionar u ordenar: por lo general se emparejan cada una de las opciones del primer bloque con las opciones dadas en el segundo bloque, o se ordenan bloques de modo que quede una secuencia correcta de acuerdo a un patrón previamente establecido. Se suelen usar en aquellos cursos donde la adquisición de conocimientos muy detallados es un objetivo importante. Son preguntas fáciles de diseñar, rápidas de calificar y se corrigen automáticamente. Estadísticamente, se tarda más en responder que las preguntas anteriores.
\item Respuesta corta: basta con que se escriban un par de palabras o una frase sencilla. Una alternativa más común a este tipo de preguntas es la de cubrir espacios en blanco con una palabra. Son de gran utilidad a la hora de demostrar los conocimientos basados en hechos o palabras claves. La dificultad para calificarlas depende del estilo que se decida emplear.
\end{itemize}

\subsection{Proceso de validación de las respuestas}
Al tratarse de un sistema de capacitación automatizado, por lo general, el método utilizado para validar las respuestas es el automático. De esta forma se facilita el trabajo para aquellos que deben evaluar a un personal abundante. Según \cite{ExpertSystem}, la manera más efectiva y eficiente de evaluar las respuestas en estos sistemas es mediante el uso de sistemas expertos.

\subsection{Evolución a través de la historia}

%%%%%%%%%%%%%%%%%%%%%%%%%%%%%%%%%
\section{Sistemas expertos}
Los sistemas expertos resuelven problemas que normalmente son solucionados por expertos humanos. Para hacerlo, necesitan acceder a una importante base de conocimiento sobre el dominio, que debe construirse de la manera más eficiente posible. Utilizan uno o más mecanismos de razonamiento, para aplicar este conocimiento a los problemas que se le proponen. Cuentan con un mecanismo para explicar a los usuarios que han confiado en ellos, lo que han hecho y cómo \cite{VonRueden2019}.

Una forma de contemplar los sistemas expertos es que representan la mayor parte de la Inteligencia Artificial (IA) aplicada. Un sistema experto en IA se define como un programa informático que tiene la capacidad de representar y razonar sobre el conocimiento \cite{SEdiagramas}.

\subsection{Componentes}
En \cite{OdhiamboOmuya2021} se comentan los diferentes componentes que integran un sistema experto. Aunque pueden contar con un número mayor, los mínimos requeridos son (Figura \ref{fig:componentesSE}):

\begin{itemize}
\item Motor de inferencia: es el corazón del sistema experto. Su cometido principal es sacar conclusiones aplicando el conocimiento a los datos. Estas conclusiones pueden estar basadas en conocimiento determinista o conocimiento probabilístico.
\item Base de conocimiento: consiste en un conjunto de objetos y un conjunto de reglas que gobiernan las relaciones entre esos objetos. La información que almacena es de naturaleza permanente y estática, es decir, no cambia de una aplicación a otra. Se debe diferenciar entre los datos y el conocimiento. El conocimiento se refiere a afirmaciones de validez general tales como reglas, distribuciones de probabilidad, entre otras. Los datos se refieren a información relacionada con una aplicación en particular.
\item Mecanismo de aprendizaje: controla el flujo del nuevo conocimiento que va del experto humano a la base de conocimiento. El sistema determina qué nuevo conocimiento se necesita, o si el conocimiento es realidad, es decir, si debe incluirse, y en caso necesario incorporar dicho conocimiento.
\item Interfaz de usuario: es la interfaz entre el sistema experto y el usuario. Para que sea efectiva debe incorporar mecanismos para mostrar y obtener información de forma sencilla y agradable.
\end{itemize}

\begin{figure}[h]
\centering
 \includegraphics[width=0.8\linewidth]{imagen/ComponentesSE.png}
 \caption{Componentes de un sistema experto.}
 \label{fig:componentesSE} 
\end{figure}

\subsection{Ventajas}
A pesar de sus desventajas, el uso de un sistema experto en cualquier ámbito social resulta favorable de diversas maneras. Según \cite{Mitchell1990}, entre sus ventajas más destacadas se pueden encontrar:

\begin{itemize}
\item No sufre de limitaciones y percances humanos, lo que lo convierte en una herramienta estable y fiable para su entorno.
\item Sus actividades son completamente replicables y siempre contesta de la misma manera a menos que se le cambie el diseño.
\item La velocidad de procesamiento es mayor al de un ser humano.
\item Pueden almacenar su conocimiento para cuando sea necesario aplicarlo.
\item Pueden ser utilizados por personas no especializadas para resolver problemas.
\item Pueden ser usados como sistema de aprendizaje.
\item Al evaluar el costo total del empleo de esta tecnología, la replicabilidad y estabilidad, asociado a la seguridad que provee, resulta una ecuación favorable, aún considerando que las inversiones iniciales pueden ser
relativamente elevadas.
\end{itemize}

\subsection{Desventajas}
A pesar de las ventajas que puede proporcionar el uso de sistemas expertos, su tecnología novedosa trae consigo ciertas desventajas. En \cite{SystemExpertsEstudents} se mencionan algunas de ellas:

\begin{itemize}
\item Para actualizarlos se necesita de reprogramación, siendo una de sus limitaciones más acentuadas.
\item Son poco flexibles a cambios y de difícil acceso a información no estructurada.
\item Poseen elevado costo en dinero y tiempo. 
\item Carecen de sentido común (no hay nada obvio).
\item No se puede mantener una conversación informal con ellos.
\item Es muy complicado que aprendan de sus errores o de errores ajenos.
\item No son capaces de distinguir cuáles son las cuestiones relevantes de un
problema y separarlas de cuestiones secundarias.
\end{itemize}

Sin embargo, estos problemas no solo los presentan los sistemas expertos. La Inteligencia Artificial (IA) aún no ha podido desarrollar sistemas que
sean capaces de resolver problemas de manera general o de aplicar el sentido común para resolver situaciones complejas. Es por ello que, a pesar de sus desventajas, los sistemas expertos son considerados una gran ayuda y un enorme avance, en especial, en el campo de los sistemas de capacitación \cite{Barham2019}.

%%%%%%%%%%%%%%%%%%%%%%%%%%%%%%%%%
\section{Sistema de entrenamiento SECPROIT}
En el año 2019 se creó el Sistema Experto para el Control de Procesos Químicos (SECPROIT), que es un sistema diseñado para capacitar a los operarios de la Industria Alimentaria Cubana sobre los diferentes procesos productivos que en ella se realizan. Este sistema fue desarrollado en la Universidad Tecnológica de La Habana José Antonio Echeverría (CUJAE), entre las facultades de Ingeniería Química e Ingeniería Informática. Su objetivo principal es lograr capacitar a los operarios de la industria y, para ello, realiza un conjunto de entrenamientos evaluados por un sistema experto \cite{BasesConocimientoArt}.

\subsection{Funcionamiento}
El sistema SECPROIT está diseñado para capacitar a los trabajadores de las fábricas, a partir de entrenamientos relacionados con los procesos productivos que en ellas se realizan, mediante el uso de sistemas expertos. En este, existen tres roles fundamentales: 

\begin{itemize}
\item Administrador: se encarga del control de los datos del sistema.
\item Especialista: es el responsable de insertar las bases de conocimiento en el sistema y supervisa los resultados obtenidos por los trabajadores de su área laboral.
\item Operario: realiza los entrenamientos.
\end{itemize}

Cada usuario tiene un nombre, una contraseña y un rol, y con cada rol aparecen funcionalidades únicas y específicas. En el caso de los especialistas, para insertar una base de conocimiento deben asociarla a un proceso, y de cada proceso deben registrar el nombre, una imagen si la posee, un fichero tipo \textsl{anm} y un archivo \textsl{drl} \cite{SECPROIT}.

Los ficheros \textsl{anm} contienen los nombres de todas las variables que influyen en el proceso, sus características, las causas que pueden hacer que entren en estado de alarma y las recomendaciones para cada una de las causas descritas. Por otra parte, el archivo \textsl{drl} contiene las reglas que enlazan las variables con sus causas y las causas con sus recomendaciones. Dichas reglas siguen una estructura específica: comienzan con un patrón para el nombre de las reglas, luego presentan los posibles atributos que poseen, las sentencias que se deben cumplir y las acciones que realizar si se cumplen las sentencias (Figura \ref{fig:drools}). Este archivo es lo que se conoce como Motor de Reglas (Drools) \cite{BaseConocimiento}.

\begin{figure}[h]
\centering
 \includegraphics[width=0.5\linewidth]{imagen/EstructuraDRL.png}
 \caption{Estructura de un archivo \textsl{drl}.}
 \label{fig:drools} 
\end{figure}

El operario entrará a realizar aquellos entrenamientos que el especialista haya validado. Una vez comenzada la prueba, deberá señalar de un grupo de variables las que por su estado estén en peligro de inestabilidad. Si ha seleccionado correctamente pasa a la siguiente etapa, donde debe escoger qué causa el estado de las variables que prefirió. Por último deberá seleccionar qué recomendaciones seguir para cada causa que señaló \cite{SECPROIT}.

Este sistema posee un grupo de reportes que facilita la toma de decisiones a partir de los resultados alcanzados. El especialista puede contar con una lista de resultados de cada operario de su área y, de esta forma, se garantiza la selección de los mejores trabajadores a partir de la organización brindada por la lista. También se puede encontrar un listado de los cambios realizados en el sistema, junto con la fecha y el nombre del usuario que lo realizó.

\subsection{Limitaciones actuales del sistema}
Actualmente el sistema SECPROIT posee ciertas limitaciones que impiden que sea implementado en las industrias:

\begin{itemize}
\item La información necesaria para generar los entrenamientos es extraída en todo momento, de forma directa, de los ficheros \textsl{anm} y \textsl{drl}, lo que genera una demora extra en el sistema.
\item Solo existe un método de pregunta.
\item El resultado de un entrenamiento está dado por la cantidad de preguntas correctas que se completaron, sin tener en cuenta el tiempo demorado.
\item La etapa final no se evalúa correctamente ni muestra la puntuación obtenida.
\item Las etapas del entrenamiento se realizan de manera continua, sin posibilidad de pausa.
\item Las etapas aparecen de forma consecutiva, sin importar si la anterior fue aprobada o no (aparecen etapas innecesariamente).
\item Por cada proceso existe un único entrenamiento, lo que limita la capacidad de aprendizaje del trabajador.
\item Las pantallas del sistema son poco intuitivas, con colores oscuros y de diversos tamaños, lo que resulta poco amigable para el usuario que las emplea.
\end{itemize}

%%%%%%%%%%%%%%%%%%%%%%%%%%%%%%%%%
\section{Generador de Bases de Conocimiento}
Como todo sistema experto, el sistema SECPROIT posee un conjunto de bases de conocimiento que, en este caso, fue creada por un Generador de Bases de Conocimiento. Dicho generador fue desarrollado en el año 2018 por la Universidad Tecnológica de La Habana José Antonio Echeverría (CUJAE), entre las facultades de Ingeniería Química e Ingeniería Informática. Es un sistema para facilitar la creación de dichas bases. En él se introducen de manera manual todos los datos relacionados a las variables, causas y recomendaciones de los procesos productivos. Luego deberán inluirse las reglas entre las variables y las causas, y las relaciones entre las causas y las recomendaciones. Una vez completada esta información, de forma semi-automática, se generan los ficheros \textsl{anm} y \textsl{drl} que serán utilizados por el SECPROIT \cite{BaseConocimiento}.

%%%%%%%%%%%%%%%%%%%%%%%%%%%%%%%%%

\section{Conclusiones Parciales}
A partir de la investigación realizada y los aspectos profundizados a lo largo del capítulo:

\begin{itemize}
\item Se hizo un estudio detallado de los sistemas de capacitación existentes, llegando a concluir que, por las condiciones que presentan, el sistema de capacitación que más se ajusta a la Industria Alimentaria Cubana es el digital.
\item Se tuvieron en cuenta todos los detalles de la problemática planteada y, gracias a las similitudes existentes, se concluyó que un sistema experto resulta la herramienta idónea para esta solución.
\item Se analizó con profundidad el sistema SECPROIT para ser tomado como base a la hora de diseñar y programar el nuevo sistema. Será el modelo a seguir y se rectificarán todos las limitaciones existentes en él para esta nueva versión.
\end{itemize}

    %\chapter*{Capítulo 2 \vspace{0.5cm} \break Desarrollo del nuevo sistema}
\setcounter{chapter}{2}
\setcounter{section}{0}
\addcontentsline{toc}{chapter}{Capítulo 2: Desarrollo del nuevo sistema}

Para que el desarrollo de un proyecto concluya con éxitos, primero debe realizarse el diseño de lo que se pretende obtener, así como analizar los requisitos que se deben cumplir. En este capítulo se presentan los diagramas utilizados en la elaboración del nuevo sistema de capacitación, así como la estructura que posee, sus principales funcionalidades y algunas comparaciones con el modelo del sistema SECPROIT.

\section{Descripción del negocio}
Lo que se pretende conseguir con este nuevo sistema es rectificar las limitaciones existentes en el SECPROIT actual. Para ello, el sistema debe satisfacer el objetivo principal del software anterior: capacitar a los operarios ante los procesos productivos de una fábrica.
Partiendo de esta premisa, la solución debe poseer un modelo de entrenamiento que ponga a prueba a los operarios de la industria. También debe contar con un administrador que sea el responsable de incluir a los usuarios y las áreas.

A la hora de realizar el entrenamiento, se deben evaluar tres etapas fundamentales: variables, causas y recomendaciones. Si el trabajador logra superar estas etapas, puede darse por aprobado el entrenamiento de ese proceso.

\subsection{Modelo de dominio}
Un modelo de dominio es la representación de las clases conceptuales del mundo real, no de componentes de software. Su utilidad radica en ser una forma de inspiración para el diseño de entidades. Es el artefacto clave en el análisis orientado a objetos \cite{Herchi2012}.

El modelo de dominio de este sistema (\textsl{Figura \ref{fig:dominio}}) no es muy diferente del modelo del sistema SECPROIT. Para representarlo se tuvieron en cuenta tres colores, con el objetivo de diferenciar los cambios realizados:

\begin{itemize}
\item \textbf{Azul}: representa aquellas entidades que no sufrieron ningún cambio significativo con respecto al sistema SECPROIT
\item \textbf{Amarillo}: representa aquellas entidades que fueron modificadas, pero que siguen cumpliendo las mismas funciones
\item \textbf{Verde}: representa las entidades que son totalmente nuevas
\end{itemize}

\begin{figure}[h]
\centering
 \includegraphics[width=0.7\linewidth]{imagen/dominio.png}
 \caption{Modelo de dominio}
 \label{fig:dominio} 
\end{figure}

Las entidades que contienen algunos cambios (\textsl{Tabla \ref{tab:ent-amarilla}}), indican que también se realizaron arreglos en las estructuras que estas seguían, es decir, variaciones en la base de datos y en las clases del sistema.

\begin{table}[H]
\begin{center}
\begin{tabular}{ | c | p{5cm} |  p{5.2cm} | }
\hline
\textbf{Entidad} & \textbf{Descripción} & \textbf{Cambios}\\
\hline
Sistema SECPROIT & Representa el sistema & Cambian las clases que posee \\
\hline
Jefe de área & Es un tipo de usuario y se encarga de administrar los procesos & Antes era especialista y solo creaba los procesos \\
\hline
Proceso & Representa los procesos & Se le agrega un nuevo atributo: configuración de entrenamiento \\
\hline
Entrenamiento & Representa el entrenamiento del usuario & Se le agrega nuevos intentos por cada etapa \\
\hline
\end{tabular}
\caption{Modelo de dominio: entidades que sufrieron cambios}
\label{tab:ent-amarilla}
\end{center}
\end{table}

La entidades que no fueron modificadas (\textsl{Tabla \ref{tab:ent-azul}}),  no sufrieron cambios porque, basadas en las tareas que desempeñan, no varían sus funciones en este nuevo modelo.

\begin{table}[H]
\begin{center}
\begin{tabular}{ | c | p{12cm} | }
\hline
\textbf{Entidad} & \textbf{Descripción} \\
\hline
Área & Representa el lugar donde laboran los trabajadores y es donde están presentes los procesos \\
\hline
Usuario & Representa a la persona que va interactuar con el sistema \\
\hline
Administrador & Es un tipo de usuario y se encarga de administrar las entidades del sistema (áreas y demás usuarios) \\
\hline
Operario & Es un tipo de usuario y se encarga de realizar los entrenamientos \\
\hline
Base de datos & Representa el archivo \textsf{anm} \\
\hline
Base de reglas & Representa el archivo \textsf{drl} \\
\hline
Drools & Es la entidad que permite ejecutar las reglas del proceso \\
\hline
Variable & Forma parte de la base de datos del proceso y contiene la información que se se evalúa en la primera etapa \\
\hline
Causa & Forma parte de la base de datos del proceso y contiene la información que se evalúa en la segunda etapa \\
\hline
Recm. & Forma parte de la base de datos del proceso, representa las recomendaciones y contiene la información que se evalúa en la tercera etapa \\
\hline
Variable-Causa & Forma parte de la base de reglas del proceso y contiene la información que se evalúa en la primera y segunda etapa \\
\hline
Causa-Recm. & Forma parte de la base de reglas del proceso y contiene la información que se evalúa en la tercera etapa \\
\hline
\end{tabular}
\caption{Modelo de dominio: entidades que no sufrieron cambios}
\label{tab:ent-azul}
\end{center}
\end{table}

Por último, las entidades nuevas (\textsl{Tabla \ref{tab:ent-verde}}) son clases que fueron incluidas con el fin de enmendar alguna limitación.

\begin{table}[H]
\begin{center}
\begin{tabular}{ | c | p{12cm} | }
\hline
\textbf{Entidad} & \textbf{Descripción} \\
\hline
Configuración & Es la configuración del entrenamiento de un proceso, es decir, un grupo de características determinadas en el entrenamiento \\
\hline
Etapas & Representa las etapas del entrenamiento \\
\hline
E-Variable & Es la primera etapa, donde se evalúan las variables \\
\hline
E-Causa & Es la segunda etapa, donde se evalúan las causas \\
\hline
E-Recm. & Es la tercera etapa, donde se evalúan las recomendaciones \\
\hline
\end{tabular}
\caption{Modelo de dominio: entidades nuevas}
\label{tab:ent-verde}
\end{center}
\end{table}

\subsection{Reglas del negocio}
Las reglas de un negocio son directrices y restricciones que ayudan a regular las operaciones de una entidad determinada. Para cada proceso existen reglas que deben seguirse durante la ejecución, ya que estas ayudan a definir \textbf{cómo} deben realizarse las tareas, por \textbf{quién}, \textbf{cuándo}, \textbf{dónde} y \textbf{por qué} \cite{Chisholm2007}. A modo de resumen, las reglas de un negocio son límites impuestos a las operaciones para que estén en sintonía con las políticas y objetivos de la institución.

En el sistema SECPROIT existen un conjunto de reglas primordiales que no se deben dejar de cumplir:
\begin{itemize}
\item Cada usuario debe poseer un único rol
\item Para poder introducir un nuevo usuario debe existir, al menos, un área laboral
\item Un usuario no puede pertenecer a más de un área
\item Solo puede existir un jefe de área por área
\item Para poder introducir un nuevo proceso debe existir, al menos, un área laboral
\item Un proceso no puede pertenecer a más de un área
\item Para generar un nuevo entrenamiento debe existir, al menos, un usuario que cumpla el rol de operario y un proceso que sea de la misma área
\item Para entrenar en la etapa de las causas debe haberse aprobado la etapa de las variables
\item Para entrenar en la etapa de las recomendaciones debe haberse aprobado la etapa de las causas
\item Los usuarios con rol de administrador son los únicos que pueden gestionar las áreas y las cuentas de los usuarios
\item No se puede eliminar la cuenta de un usuario
\item Los usuarios con rol de jefe de área son los únicos que pueden gestionar los procesos de sus áreas y configurar los entrenamientos
\item Los usuarios con rol de operario son los únicos que pueden entrenar
\item Para superar un entrenamiento se deben haber aprobado las tres etapas (variables, causas y recomendaciones)
\end{itemize}

\subsection{Diagrama de actividades}
El Lenguaje Unificado de Modelado (UML) incluye varios subconjuntos de modelos, entre los que se encuentran los diagramas de actividades. Estos diagramas son considerados diagramas de comportamiento, porque describen el comportamiento del sistema que representan \cite{Eriksson2000}.

En el diagrama de actividades del nuevo sistema de capacitación (\textsl{Figura \ref{fig:actividades}}) se pueden observar tres colores: azul, amarillo y verde. El color azul indica que el flujo funciona de la misma manera que en el SECPROIT, el color amarillo representa un ligero cambio y el verde, representa una acción totalmente nueva.

\begin{figure}[h]
\centering
 \includegraphics[width=0.65\linewidth]{imagen/actividades.png}
 \caption{Diagrama de actividades}
 \label{fig:actividades} 
\end{figure} 

\section{Captura de requisitos}
La extracción o captura de los requisitos en un sistema es una de las fases más críticas e importantes en el desarrollo de software. Esta fase tiene como objetivo descubrir y recoger todas las condiciones funcionales y no funcionales de una aplicación. La actividad de descubrimiento es una tarea más humana que técnica, ya que la mayor parte de las veces los usuarios no serán capaces de definir todas las condiciones \cite{Dave2022}. 

Para el desarrollo de este sistema se realizó un estudio de requisitos bastante extensivo. Se coordinó con los interesados y se acordó una lista de requerimientos funcionales y no funcionales.

\subsection{Requisitos funcionales}
Los requisitos funcionales son las declaraciones de los servicios que prestará el sistema. Cuando hablamos de entradas, no necesariamente hablamos sólo de los usuarios, pueden ser: interacciones con otros sistemas, respuestas automáticas, procesos predefinidos, entre otros. En algunos casos, los requisitos funcionales también establecen explícitamente lo que el sistema no debe hacer \cite{Dave2022}.

Entre los requisitos funcionales que debe poseer el nuevo sistema de capacitación, se deben incluir los cambios presentados en el modelo de dominio. Como resultado, los requisitos funcionales del nuevo sistema son:

\begin{itemize}
\item \textbf{Iniciar sesión en el sistema}: es una acción que todos los usuarios pueden realizar.
\item \textbf{Cambiar contraseña personal}: es una acción que todos los usuarios pueden realizar.
\item \textbf{Gestionar usuarios}: es una actividad desarrollada solo por los administradores. Se basa en introducir, modificar y desactivar o activar los usuarios del sistema, aunque también cuenta con una opción para restablecer la contraseña en caso de que el usuario la haya olvidado.
\item \textbf{Gestionar áreas}: es una actividad desarrollada solo por los administradores. Se basa en introducir, modificar y eliminar las áreas del sistema. Para poder eliminar un área laboral, esta debe estar vacía, es decir, que de ella no dependa ningún usuario.
\item \textbf{Ver acciones de los usuarios}: es una actividad desarrollada solo por los administradores. Permite observar, en una tabla, todas las actividades realizadas por los usuarios.

\item \textbf{Gestionar procesos}: es una actividad desarrollada solo por los jefes de área. Se basa en introducir, modificar y eliminar los procesos en un área laboral. Si se elimina un proceso, queda registro del mismo y los entrenamientos que se hayan realizado no se pierden.
\item \textbf{Configurar entrenamiento}: es una actividad desarrollada solo por los jefes de área. Consiste en asignar para cada proceso los estilos de pregunta que se pueden aplicar, la cantidad general de intentos, el tiempo máximo para realizar el entrenamiento y los usuarios que pueden evaluarse.
\item \textbf{Ver resultados de los operarios del área}: es una actividad desarrollada solo por los jefe de área. Permite observar, en una tabla, todos los resultados de los operarios que pertenezcan a esa área.

\item \textbf{Entrenamiento}: es una acción desarrollada solo por los operarios. Consiste en evaluarse sobre un proceso productivo. Se deben responder un conjunto de preguntas por etapas para luego obtener un resultado general que será registrado en el sistema.
\item \textbf{Ver resultados}: es una actividad desarrollada solo por los operarios. Permite observar, en una tabla, todos los resultados obtenidos en los entrenamientos.
\end{itemize}

\subsection{Requisitos No Funcionales}
Los requisitos no funcionales son condiciones que no se refieren directamente a las funciones específicas suministradas por un sistema (características de usuario), sino a las propiedades del mismo: rendimiento, seguridad, disponibilidad, entre otros. En palabras más sencillas, no hablan de lo que hace el sistema, sino de cómo lo hace. Alternativamente, definen restricciones del sistema tales como la capacidad de los dispositivos de entrada/salida y la representación de los datos utilizados en la interfaz del sistema \cite{Dave2022}.

En este caso, los requisitos no funcionales presentes en este nuevo sistema son los mismos del SECPROIT:

\begin{itemize}
\item \textbf{Usabilidad}: se debe garantizar un ambiente de trabajo simple e intuitivo, ya que la mayoría de los usuarios no poseen experiencias con sistemas informáticos
\item \textbf{Seguridad}: la información del sistema solo puede ser manipulada por usuarios autorizados (aquellos que posean usuario y contraseña)
\item \textbf{Confiabilidad}: se deben evitar los enlaces rotos, los ficheros de los procesos deben ser validados antes de utilizarlos y se debe garantizar la confidencialidad de la información
\item \textbf{Disponibilidad}: la aplicación de mecanismos de seguridad no debe constituir un retraso para el uso del sistema, el software siempre debe estar disponible, así como brindar su información actualizada
\item \textbf{Software}: se debe tener instalado el JDK versión 1.8 y la aplicación PostgreSQL versión 9.1 (mínimo) para el manejo de la base de datos
\item \textbf{Hardware}: se necesitan 64MB de memoria RAM, un microprocesador Pentium II a 450 MHz (mínimo), un disco duro con capacidad libre (mínima) de 4GB y un sistema operativo de entorno gráfico como Windows y Linux
\item \textbf{Portabilidad}: debe ser utilizado bajo sistemas operativos Windows o Linux, por lo que su desarrollo debe realizarse con un lenguaje y tecnologías capaces de brindar este soporte
\item \textbf{Restricciones en el diseño y la implementación}: debe desarrollarse sobre plataformas de software libre y código abierto y su lenguaje de programación debe ser Java, debido al uso de la herramienta \textsl{Drools}
\item \textbf{Políticos/Culturales}: debe encontrarse en idioma español
\end{itemize}

%%%%%%%%%%%%%%%%%%%%%%%%%%%%%%%%%%%%%%%%%

\section{Casos de uso}
Un caso de uso representa una unidad funcional coherente en un sistema, subsistema o clase. En ellos, uno o más actores interaccionan con las entidades que realizan las acciones. El modelado de estos casos de uso permite que los desarrolladores de un software y los clientes lleguen a un acuerdo sobre los requisitos y posibilidades que debe cumplir el sistema \cite{Kalaivani2004}.

\subsection{Actores del sistema} 
Un actor puede referirse a cualquier cosa que interaccione con el sistema y que es externo a él. No necesariamente coinciden con los usuarios. Un usuario puede interpretar distintos roles, correspondientes a distintos actores. Un actor puede desempeñar distintos papeles dependiendo del caso de uso en que participe \cite{Kalaivani2004}.

El Sistema Generador de Bases de Conocimiento (SGBC) es el encargado de confeccionar los ficheros que son registrados en los procesos dentro del sistema SECPROIT, como ya se explicó en el capítulo anterior. Sin embargo, este generador no se incluye como actor en el SECPROIT, ya que no interactúa directamente con el sistema, su relación es a partir de los ficheros que genera.

Los actores presentes en esta solución (\textsl{Tabla \ref{tab:actores}}) están representados por los tres roles que puede presentar un usuario: administrador, jefe de área u operario.

\begin{table}[H]
\begin{center}
\begin{tabular}{ | c | p{11cm} | }
\hline
\textbf{Actor} & \textbf{Descripción} \\
\hline
Usuario & Actor genérico que hace uso de las funcionalidades que son comunes (iniciar sesión y cambiar contraseña personal) \\
\hline
Operario & Actor que se capacita mediante los entrenamientos generados de cada proceso y tiene acceso a sus resultados \\
\hline
Jefe de área & Actor que vincula los procesos en las áreas, los configura, decidiendo cómo será el entrenamiento, qué usuarios podrán realizarlo y las características que deberán cumplir para aprobarlo, y tiene acceso a los resultados de los operarios de su área (antes era conocido como especialista) \\
\hline
Administrador & Actor que gestiona las áreas y los usuarios y tiene acceso a un resumen de acciones de cada usuario \\
\hline
\end{tabular}
\caption{Actores del sistema}
\label{tab:actores}
\end{center}
\end{table}

\subsection{Diagrama de casos de uso}
Los diagramas de casos de uso muestran las relaciones entre las acciones de un sistema y sus actores. Estos modelos dan sólo una visión general y ayudan a interpretar y esclarecer los casos de uso \cite{Kalaivani2004}.

El diagrama de casos de uso de la solución (\textsl{Figura \ref{fig:dcu}}) contiene tres colores diferentes: azul para representar los casos de uso que no han sido modificados con respecto al modelo anterior, amarillo para representar los que sufrieron algún cambio significativo y verde para los casos de uso nuevos.

\begin{figure}[h]
\centering
 \includegraphics[width=0.7\linewidth]{imagen/dcu.png}
 \caption{Diagrama de casos de uso}
 \label{fig:dcu} 
\end{figure}

\subsection{Especificación de los casos de uso}

\begin{table}[H]
\begin{center}
\begin{tabular}{ | c | p{3.5cm} |  p{7.5cm} |}
\hline
\textbf{Actor} & \textbf{Caso de uso} & \textbf{Descripción}\\
\hline
Usuario & Iniciar sesión en el sistema & El actor debe introducir su nombre y su contraseña para iniciar en el sistema \\
\cline{2-3}
& Cambiar contraseña personal  & El actor puede cambiar su propia contraseña \\
\hline
Administrador & Gestionar áreas & El actor puede introducir, modificar o eliminar las áreas del sistema \\
\cline{2-3}
& Gestionar usuarios & El actor puede introducir, modificar, restablecer la contraseña, dormir o activar a los usuarios del sistema \\
\cline{2-3}
& Ver acciones de los usuarios & El actor puede visualizar un registro con todas las acciones que los usuarios han realizado en el sistema \\
\hline
Operario & Entrenar & El actor inicia el entrenamiento de un proceso \\
\cline{2-3}
& Evaluar etapa 1 & El actor se evalúa mediante preguntas de variables, decidiendo cuáles están fuera de rango y cuáles no \\
\cline{2-3}
& Evaluar etapa 2 & El actor se evalúa mediante preguntas de causas, decidiendo cuáles se relacionan a un conjunto de variables \\
\cline{2-3}
& Evaluar etapa 3 & El actor se evalúa mediante preguntas de recomendaciones, decidiendo cuáles se relacionan a un conjunto de causas \\
\cline{2-3}
& Ver resultados & El actor puede visualizar un registro con todos los resultados que ha obtenido \\
\hline
Jefe de área & Gestionar procesos & El actor puede introducir, modificar o eliminar los procesos de su área \\
\cline{2-3}
& Configurar entrenamiento & El actor decide para cada proceso cómo será el entrenamiento \\
\cline{2-3}
& Añadir usuarios & El actor decide para cada proceso los usuarios que pueden participar \\
\cline{2-3}
& Ver resultados de los operarios del área & El actor puede visualizar un registro con todas las evaluaciones de los operarios de su área \\
\hline
\end{tabular}
\caption{Especificación de los casos de uso}
\label{tab:caso-uso}
\end{center}
\end{table}

%%%%%%%%%%%%%%%%%%%%%%%%%%%%%%%%%%%%%%%%%%

\section{Modelo de datos}
Para el desarrollo de este sistema se utiliza, como gestor de base de datos, la herramienta PostgreSQL. En ella se almacena toda la información con la que se va a trabajar, incluyendo los ficheros \textsl{anm} y \textsl{drl} de los procesos. Por momentos determinados, esta información también se encuentra almacenada, de forma local y temporal, en clases del propio sistema, lo que permite un mejor manejo y control de los datos.

\subsection{Diagrama de Entidades y Relaciones (DER)}
Un Diagrama de Entidades y Relaciones (DER) es una herramienta que permite representar de manera simplificada los componentes que participan en un proceso, y el modo en el que estos se relacionan entre sí. Posee tres elementos principales: las entidades, los atributos y las relaciones. %cita aquí

El diagrama de entidades de este sistema (\textsl{Figura \ref{fig:der}}) contiene varios cambios con respecto al diagrama presentado en el sistema SECPROIT original. Para poder representarlos se tuvieron en cuenta una serie de colores a modo de guía:

\begin{itemize}
\item \textbf{Azul}: representa aquellas entidades que no sufrieron ningún cambio significativo con respecto al sistema anterior
\item \textbf{Amarillo}: representa aquellas entidades que fueron modificadas, pero que siguen cumpliendo las mismas funciones
\item \textbf{Verde}: representa las entidades que son totalmente nuevas, que no existen el sistema anterior
\end{itemize}

\begin{figure}[h]
\centering
 \includegraphics[width=0.6\linewidth]{imagen/der.png}
 \caption{Diagrama de Entidades y Relaciones (DER)}
 \label{fig:der} 
\end{figure}

Con el fin de realizar una compración más detallada, en \cite{ElenaAcostaGil2018} se encuentra el DER del sistema original. En este nuevo diagrama no solo se cambiaron y se agregaron nuevas entidades, también se eliminaron algunas, debido a que ya no eran necesarias su funciones. Los cambios, explicados en la \textsl{Tabla \ref{tab:entidades}}, fueron realizados con el fin de resolver algunas de las limitaciones existentes en el sistema SECPROIT.

\begin{table}[H]
\begin{center}
\begin{tabular}{ | c | p{5cm} |  p{5cm} | }
\hline
\textbf{Entidad} & \textbf{Atributos}  & \textbf{Descripción}\\
\hline
Área & ID y nombre & No sufrió ningún cambio \\
\hline
Usuario & ID, nombre, carnet, nivel, experiencia, años como jefe, nombre de usuario, contraseña, área, rol y si está activo o no & Se agregaron los atributos: nivel escolar, experiencia, años como jefe y si está activo o no el sistema\\
\hline
Traza & ID, usuario, acción y fecha & No sufrió ningún cambio \\
\hline
Proceso & ID, nombre, foto del proceso, fichero de datos, fichero de reglas y área & No sufrió ningún cambio \\
\hline
Variable & ID, nombre, tipo y proceso & Se agregó para agilizar el proceso de lectura \\
\hline
Causa & ID y nombre & Se agregó para agilizar el proceso de lectura \\
\hline
Recomendación & ID y nombre & Se agregó para agilizar el proceso de lectura \\
\hline
Variable-Causa & ID, variable y causa & Se agregó para agilizar el proceso de lectura \\
\hline
Causa-Recomendación & ID, causa y recomendación & Se agregó para agilizar el proceso de lectura \\
\hline
Configuración & ID, proceso, tiempo límite, preguntas límites y tipos de pregunta & Se agregó con el fin de poder configurar los entrenamientos \\
\hline
Entrenamiento & ID, usuario, configuración de proceso, cantidad de intentos y nota & Se eliminaron los demás atributos que poseía \\
\hline
Etapa-Entrenamiento & ID, entrenamiento, tipo de etapa, tiempo demorado, preguntas acertadas, preguntas totales y nota & Se agregó para poder tener una pausa entre etapas y poder realizar más de una prueba por entrenamiento \\
\hline
\end{tabular}
\caption{Entidades del sistema}
\label{tab:entidades}
\end{center}
\end{table}

\subsection{Diagrama de base de datos}
	%\chapter*{Capítulo 3 \vspace{0.5cm} \break Validación del nuevo sistema de capacitación}
\setcounter{chapter}{3}
\setcounter{section}{0}
\addcontentsline{toc}{chapter}{Capítulo 3: Validación del nuevo sistema de capacitación}

Partiendo del objetivo principal de esta investigación, rectificar las limitaciones existentes en el sistema SECPROIT, se desarrollaron un conjunto de pruebas y experimentos para demostrar que se enmendaron dichas restricciones. En el siguiente capítulo se presentan los resultados obtenidos en cada una de las pruebas. Este proceso parte de cómo la fase de análisis y diseño se unen para llevar a cabo el sistema propuesto.

\section{Pruebas funcionales}
 Las pruebas de software, o pruebas funcionales, son el proceso de ejecutar un sistema o componente para medir su calidad, con la intención de encontrar errores que aún no se descubren \cite{Buehler2008}.

Para este software, el nivel de prueba que se utiliza es: pruebas de sistema. Consiste en ejecutar el sistema completo, buscando defectos tanto en aspectos generales como en particulares. Se utilizan pruebas basadas en las funcionalidades (pruebas funcionales) y el método que se emplea es el de caja negra (se lleva a cabo sobre la interfaz del software) \cite{Nidhra2012}.

\subsection{Casos de prueba para la configuración entrenamientos}
Como se pudo apreciar en el capítulo 2, en el sistema desarrollado se introdujeron nuevas entidades, entre ellas: la configuración de un entrenamiento. Con esta configuración el jefe de área puede decidir el tiempo que puede tomar completar el entrenamiento de una etapa, el tipo de preguntas que pueden aparecer, la cantidad de intentos totales permitidos por cada etapa y la cantidad de intentos aprobados necesarios para superar cada una.

Gracias a esta nueva configuración, se lograron resolver algunas de las limitaciones que se mencionaban en el capítulo 1: 
\begin{itemize}
\item Existe más de un modelo de pregunta, solucionando el problema de escasez en el método de aprendizaje
\item Las etapas no aparecen de forma continúa, porque para poder avanzar a la siguiente se debe aprobar la etapa actual un número determinado de veces
\item Solo aparece una etapa a la vez
\item Para cada etapa existe más de un entrenamiento
\end{itemize}

Sin embargo, es necesario probar el correcto funcionamiento de este nueva entidad. Con ese fin, se diseñaron tres casos de prueba  (\textsl{Tabla \ref{cas1}}, \textsl{Tabla \ref{cas2}}, \textsl{Tabla \ref{cas3}}), cada una con sus propios experimentos. 

\subsubsection{Casos de prueba: Insertar configuración de entrenamiento}
El proceso de crear una nueva configuración de entrenamiento se realiza al insertar un nuevo proceso (todo ocurre en la misma interfaz de usuario). Para este caso de prueba se realizaron tres experimentos:

\begin{table}[H]
\caption{Casos de prueba: Insertar configuración de entrenamiento}
\begin{center}
\begin{tabular}{ | p{4cm} | p{3.8cm} | p{3.1cm} | p{3.2cm} |}
\hline
\centering\textbf{Prueba} & \textbf{Descripción} & \textbf{Resultado \break esperado} & \textbf{Resultado \break obtenido} \\
\hline
Insertar configuración con campos en blanco & Una configuración de entrenamiento no puede contener información vacía & Los campos están llenos por defecto y la interfaz no debe permitir que se vacíen & La interfaz no permitió que los campos fueran vaciados (Anexo \ref{fig:cb}) \\
\hline
Insertar configuración con campos incorrectos & Para introducir información incorrecta se debe insertar un número de intentos verídicos mayor que el número de intentos & Debe aparecer un mensaje de error & Aparece un mensaje de error (Anexo \ref{fig:ci}) \\
\hline
Insertar configuración con campos correctos & Introducir campos correctos & Se debe introducir la configuración  & Se introdujo la nueva configuración (Anexo \ref{fig:cc}) \\
\hline
\end{tabular}
\label{cas1}
\end{center}
\end{table}

\subsubsection{Casos de prueba: Modificar configuración de entrenamiento}
El proceso de modificar la configuración de un entrenamiento se realiza al modificar un proceso (todo ocurre en la misma interfaz). Para este caso de prueba se realizaron tres experimentos:

\begin{table}[H]
\caption{Casos de prueba: Modificar configuración de entrenamiento}
\begin{center}
\begin{tabular}{ | p{4cm} | p{4cm} | p{3.1cm} | p{3cm} |}
\hline
\centering\textbf{Prueba} & \textbf{Descripción} & \textbf{Resultado \break esperado} & \textbf{Resultado \break obtenido} \\
\hline
Modificar configuración con campos en blanco & Una configuración de entrenamiento no puede contener información vacía & Los campos están llenos por defecto y la interfaz no debe permitir que se vacíen los campos & La interfaz no permitió que los campos fueran vaciados \\
\hline
Modificar configuración con campos incorrectos & La única forma de introducir información incorrecta es insertando un número de intentos verídicos mayor que el número de intentos total & Debe aparecer un mensaje de error & Aparece un mensaje de error \\
\hline
Modificar configuración con campos correctos & Introducir campos correctos & Se debe modificar la configuración  & Se modificó la configuración \\
\hline
\end{tabular}
\label{cas2}
\end{center}
\end{table}

\subsubsection{Casos de prueba: Eliminar configuración de entrenamiento}
Como medida de seguridad, el proceso de eliminar la configuración de un entrenamiento no existe. Para eliminarla, debe eliminarse el proceso al que esta pertenece. Por lo tanto, el único experimento que se puede realizar es: eliminar un proceso.

\begin{table}[H]
\caption{Casos de prueba: Eliminar configuración de entrenamiento}
\begin{center}
\begin{tabular}{ | c | p{4.7cm} | p{3.1cm} | p{4.1cm} |}
\hline
\centering\textbf{Prueba} & \textbf{Descripción} & \textbf{Resultado \break esperado} & \textbf{Resultado \break obtenido} \\
\hline
Eliminar & Se selecciona el proceso & Se debe eliminar & Se eliminó (Anexo \ref{fig:ce}) \\
\hline
\end{tabular}
\label{cas3}
\end{center}
\end{table}

\subsection{Casos de prueba para el entrenamiento en la etapa de las causas}
En el sistema SECPROIT, la etapa de las causas no se evalúa correctamente. Si la variable presenta más de una causa, el sistema evalúa la primera pero no logra evaluar las demás. En cambio, si del grupo de variables solo una se encuentra fuera de rango, la etapa de las causas no se evalúa en absoluto.

Sin embargo, el nuevo sistema desarrollado no presenta esta dificultad. En el software, para cada tipo de pregunta en la etapa de las causas, se evalúan un número distinto de variables (todas fuera de rango). Por ejemplo, en las preguntas de completar los espacios en blanco se evalúan cinco variables distintas (todas fuera de rango) y de cada una se preguntan sus causas (Anexo \ref{fig:pregcaus}}), mientras que en las preguntas de enlazar, solo se pregunta por una variable, pero esta posee múltiples causas.

Para probar el correcto funcionamiento del entrenamiento en la etapa de las causas, se evaluaron cuatro entrenamientos de esta etapa (\textsl{Tabla \ref{cas:causa}}), uno por cada pregunta (verdadero o falso, completar los espacios en blanco, selección múltiple y enlazar). Como las preguntas que se generan en los entrenamientos son aleatorias, para estos casos de prueba el valor de datos de entrada no es relevante. Para afirmar que las respuestas son correctas, se establece una comparación entre las respuestas dadas por el sistema y las respuestas que brinda el motor de reglas (\textsl{JDrools}).

\begin{table}[H]
\caption{Casos de prueba para el entrenamiento en la etapa de las causas}
\begin{center}
\begin{tabular}{ | p{3cm} | p{5.9cm} | p{2.6cm} | p{2.6cm} |}
\hline
\centering\textbf{Prueba} & \textbf{Descripción} & \textbf{Resultado del JDrools} & \textbf{Resultado del sistema} \\
\hline
Verdadero o \break falso & Se presentan 5 variables con sus causas y se debe indicar si la afirmación es verdadera o falsa & Se acertaron 2 afirmaciones &  Se acertaron 2 afirmaciones \\
\hline
Completar los espacios en blanco & Se presentan 5 variables y de 1 a 5 causas, y se debe rellenar el espacio en blanco de manera que quede una expresión verdadera & Se acertaron 3 expresiones &  Se acertaron 3 expresiones (Anexo \ref{fig:causres}) \\
\hline
Enlazar & Se presenta 1 variable y 5 causas, y se deben enlazar las que correspondan a esa variable & Se acertaron 4 &  Se acertaron 4 \\
\hline
Selección múltiple & Se presenta 1 variable y 5 causas, y se deben seleccionar las que correspondan a esa variable & Se acertó 1 selección &  Se acertó 1 selección \\
\hline
\end{tabular}
\label{cas:causa}
\end{center}
\end{table}

\subsection{Casos de prueba para el entrenamiento en la etapa de las recomendaciones}
En el sistema SECPROIT, la etapa de las recomendaciones no se evalúa. Al concluir la evaluación de la etapa de las causas, el sistema califica la etapa de las recomendaciones de excelente y se acaba el entrenamiento (Anexo \ref{fig:errorRec}).

Sin embargo, en el nuevo sistema desarrollado si se evalúan las recomendaciones. Para cada tipo de pregunta se evalúan un número distinto de recomendaciones. Por ejemplo, en las preguntas de verdadero o falso se evalúan cinco recomendaciones, mientras que en las preguntas de enlazar, se evalúan de una a cinco.

Para probar el correcto funcionamiento del entrenamiento en la etapa de las recomendaciones, se evaluaron cuatro entrenamientos de esta etapa (\textsl{Tabla \ref{cas:reco}}), uno por cada pregunta. Como las preguntas que se generan en los entrenamientos son aleatorias, para estos casos de prueba el valor de datos de entrada no es relevante. Para afirmar que las respuestas son correctas, se establece una comparación entre las respuestas dadas por el sistema y las respuestas que brinda el motor de reglas (\textsl{JDrools}).

\begin{table}[H]
\caption{Casos de prueba para el entrenamiento en la etapa de las recomendaciones}
\begin{center}
\begin{tabular}{ | p{2.7cm} | p{6.2cm} | p{2.6cm} | p{2.6cm} |}
\hline
\centering\textbf{Prueba} & \textbf{Descripción} & \textbf{Resultado del JDrools} & \textbf{Resultado del sistema} \\
\hline
Verdadero o \break falso & Se presentan 5 causas con sus recomendaciones y se debe indicar si la afirmación es verdadera o falsa & Se acertó 1 afirmación &  Se acertó 1 afirmación \\
\hline
Completar los espacios en blanco & Se presentan 5 causas y de 1 a 5 recomendaciones, y se debe rellenar el espacio en blanco de manera que quede una expresión verdadera & Se acertaron 3 expresiones &  Se acertaron 3 expresiones \\
\hline
Enlazar & Se presenta 1 causa y 5 recomendaciones, y se deben enlazar las que correspondan a esa causa & Se acertaron 2 &  Se acertaron 2 \\
\hline
Selección múltiple & Se presentan 1 causa y 5 recomendaciones, y se deben seleccionar las que correspondan a esa causa & Se acertaron 2 selecciones &  Se acertaron 2 selecciones \\
\hline
\end{tabular}
\label{cas:reco}
\end{center}
\end{table}

\section{Pruebas de rendimiento}
Las pruebas de rendimiento son una técnica de prueba de software no funcional que determina cómo la estabilidad, la velocidad, la escalabilidad y la capacidad de respuesta de una aplicación se mantiene bajo una determinada carga de trabajo. Es un paso clave para asegurar la calidad del software. Los objetivos de estas pruebas incluyen la evaluación de la salida de la aplicación, la velocidad de procesamiento, la velocidad de transferencia de datos, el uso del ancho de banda de la red, el máximo de usuarios concurrentes, la utilización de la memoria, la eficiencia de la carga de trabajo y los tiempos de respuesta de los comandos \cite{Medina2014}.

\subsection{Proceso de entrenamiento}
En el sistema SECPROIT, para generar un entrenamiento, se extraen y se leen los ficheros del proceso al que pertenece, almacenados en la base de datos. A partir de la información extraída, se generan las preguntas del entrenamiento. Para calificar las pruebas, se vuelven a extraer los ficheros de la base de datos y se vuelve a leer la información escrita en ellos. Realizar el proceso de esta forma, genera una demora extra en la ejecución del software.

En el nuevo sistema de capacitación, se decidió almacenar la información de los ficheros de los procesos, en nuevas tablas incorporadas en la base de datos (como ya se comentó en capítulos anteriores). Cuando se va a generar un entrenamiento, se extrae la información necesaria de las tablas, y se generan las preguntas. A la hora de evaluar el entrenamiento, se utiliza la misma información extraída anteriormente. Realizar el proceso de esta manera reduce el tiempo de ejecución del mismo.

Para probar que el tiempo de ejecución del proceso de entrenamiento es menor en el nuevo sistema de capacitación desarrollado, se utilizaron las pruebas de rango con signo de Wilcoxon \cite{Turcios2015}. Cabe señalar que:

\begin{itemize}
\item Ambos sistemas se ejecutaron en la misma máquina, con las mismas características
\item Cuando se realizaron las pruebas, solo se ejecutaba el sistema evaluado (no existía ningún otro proceso ejecutándose)
\item Ambas pruebas fueron cronometradas dentro del propio sistema para evitar errores humanos
\item Los ficheros de los procesos utilizados para las pruebas, son los mismos en ambos sistemas
\end{itemize}

Con esta prueba, se desea conocer la efectividad del método implementado en el nuevo sistema de capacitación para la lectura de la información de los ficheros de un nuevo entrenamiento, y para ello se toma una muestra aleatoria de 10 procesos del sistema de capacitación. Se registró el tiempo que demoró generar y evaluar un entrenamiento en el sistema SECPROIT y el tiempo que demoró en el nuevo sistema de capacitación. El tiempo se encuentra registrado en minutos. Los datos obtenidos se pueden observar en la Tabla \ref{pruebasConSigno}.

\begin{table}[H]
\caption{Comparando el tiempo demorado en el entrenamiento de un proceso}
\begin{center}
\begin{tabular}{ | c | c | c | }
\hline
\centering & \textbf{Sistema SECPROIT} & \textbf{Nuevo sistema de capacitación} \\
\centering & \textbf{(tiempo en minutos)} & \textbf{(tiempo en minutos)} \\
\hline
 Proceso #1 & 5:09 & 0:51 \\
\hline
 Proceso #2 & 5:23 & 0:55 \\
\hline
 Proceso #3 & 6:15 & 1:05 \\
\hline
 Proceso #4 & 6:01 & 1:00 \\
\hline
 Proceso #5 & 5:09 & 0:50 \\
\hline
 Proceso #6 & 5:15 & 0:52 \\
\hline
 Proceso #7 & 5:29 & 0:56 \\
\hline
 Proceso #8 & 5:23 & 0:54 \\
\hline
 Proceso #9 & 4:09 & 0:33 \\
\hline
 Proceso #10 & 5:21 & 0:50 \\
\hline
\end{tabular}
\label{pruebasConSigno}
\end{center}
\end{table}

Las hipótesis de esta prueba son: no hay diferencia en el proceso de entrenamiento entre un sistema y otro (\textbf{H}0), y el tiempo de ejecución del nuevo sistema de capacitación es menor que el del sistema SECPROIT (\textbf{H}1). Se procede a calcular su solución utilizando la herramienta R, para la función \textsl{wilcox.test()}:

\begin{figure}[H]
\centering
\includegraphics[width=0.8\linewidth]{imagen/erre.png}
 \caption{Resultados con la herramienta R}
 \label{fig:herrR} 
\end{figure}

El estadístico de prueba es 55 y el valor \textbf{p} correspondiente es 0,0009766. Dado que este valor \textbf{p} es menor que 0,05, se rechaza la hipótesis nula (\textbf{H}0) y se acepta la alternativa que el tiempo de antes es mayor que el tiempo de después. Por lo tanto, se puede afirmar que el nuevo método incorporado es más rápido que el método anterior utilizado.

\section{Conclusiones parciales}
Al concluir este capítulo se pueden arribar las siguientes conclusiones:
\begin{itemize}
\item Se logró resolver la limitación existente en el proceso de entrenamiento de la etapa de las causas
\item Se logró resolver la limitación existente en el proceso de entrenamiento de la etapa de las recomendaciones
\item Se agregaron configuraciones al entrenamiento que permiten un mejor control de los mismos y una mejor evaluación
\item Se implementó un nuevo método para extraer la información de los ficheros de los procesos que resulta más rápido que el método anterior
\end{itemize}
	    
    \cleardoublepage
  \phantomsection
	\addcontentsline{toc}{chapter}{Conclusiones}
    
 % \chapter*{Conclusiones}
Al comenzar este trabajo se propuso una serie de objetivos específicos que ayudarían a completar el objetivo general del mismo. A lo largo, se ha ido completando cada uno de esos objetivos.
Tras cumplir satisfactoriamente con los objetivos específicos, se puede llegar a la conclusión de que se cumplió con el objetivo principal de la investigación. 

    
   \cleardoublepage
   \phantomsection
	\addcontentsline{toc}{chapter}{Recomendaciones}
	
	\chapter*{Recomendaciones}
A pesar de que se lograron cumplir los objetivos de esta investigación, aún existen algunas funciones que se pueden incorporar. Algunas de estas nuevas funcionalidades son propuestas como recomendaciones:

\begin{itemize}
\item Permitir la modificación de los niveles escolares para que el usuario pueda incluir nuevos niveles
\item Desarrollar una nueva sección para gestionar los errores del sistema, donde el usuario pueda decidir que información se mostrará en cada mensaje de error
\item Añadir una conexión manual del sistema con su base de datos, donde se especifiquen los parámetros de conexión de la misma y de esta manera, no tendría que realizarse con parámetros específicos
%%%%%%%%%%%%%%%%%%%%%%%%%%%%%%%%%%%%%%%%%%
\item Insertar un paginado en las tablas
\item Validar la entrada del carnet de identidad por sus 11 dígitos
\item Desarrollar una función que permita exportar los reportes
\item Insertar un nuevo reporte basado en gráficas
\end{itemize}
	
	\cleardoublepage
	\phantomsection
	\addcontentsline{toc}{chapter}{Referencias bibliográficas}
	
	\nocite{*}
    \bibliographystyle{ieeetr}
	
	\bibliography{referencias/Referencias}
	\breakpage
	
	\phantomsection
	\addcontentsline{toc}{chapter}{Anexos}
	\pagestyle{fancy}
	
	\setcounter{section}{1}
\appendix
\clearpage{\renewcommand{\appendixname}{Anexo}

\chapter{Pruebas funcionales del sistema}

\section{Insertar configuración de entrenamiento}
\begin{figure}[H]
\centering
 \frame{\includegraphics[width=0.9\linewidth]{imagen/anexos/nuevaConf.png}}
 \caption{Insertar configuración con campos en blanco}
 \label{fig:cb} 
\end{figure}

\begin{figure}[H]
\centering
 \frame{\includegraphics[width=0.9\linewidth]{imagen/anexos/errorConf.png}}
 \caption{Insertar configuración con campos incorrectos}
 \label{fig:ci} 
\end{figure}

\begin{figure}[H]
\centering
 \frame{\includegraphics[width=0.9\linewidth]{imagen/anexos/guardarConf.png}}
 \caption{Insertar configuración con campos correctos}
 \label{fig:cc} 
\end{figure}

\section{Eliminar configuración de entrenamiento}
\begin{figure}[H]
\centering
 \frame{\includegraphics[width=0.9\linewidth]{imagen/anexos/eliminarConf.png}}
 \caption{Eliminar configuración de entrenamiento}
 \label{fig:ce} 
\end{figure}
	
\end{document}