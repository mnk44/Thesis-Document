\section*{Abstract}
The Food Industry in Cuba plays a very important role in the economy, so a problem in this sector should be avoided at all costs. Currently it presents the need to train all the operators of its plants, before the productive processes that are developed in them, since there are many irregularities when operating these processes. When there is a fault, experts are not always present for its possible solution, since not all technicians are prepared to make the correct decisions in certain circumstances. This is why, when a failure occurs, the production process stops until the expert appears, causing losses in numerous areas and affecting the economy.
As a result of this situation, the Food Industry Research Institute (IIIA), together with the Faculty of Computer Engineering and Chemical Engineering of the CUJAE, developed an Expert System for the Control of Chemical Processes (SECPROIT) and a Database Generator of Knowledge.

The SECPROIT aims to evaluate the decisions made by the operators in different critical situations and provide the area manager with the results of the evaluations and how they are overcome in each process. At this time, the system has certain limitations that prevent it from working properly, so it is necessary to update it.

\begin{description}
	\item[Keywords:]{Food Industry, production process, operator, Research Institute of the Food Industry (IIIA), Expert System for the Control of Chemical Processes (SECPROIT), Generator of Knowledge Bases.}
\end{description}