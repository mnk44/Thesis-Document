\section*{Abstract}
The Food Industry in Cuba plays a very important role in the economy, so a problem in this sector should be avoided at all costs. It currently presents the need to train all its operators in the production processes that are developed, since it consists of many irregularities when working on them. When there is a fault, experts are not always present for its possible solution, since not all technicians are prepared to make the correct decisions in certain circumstances. In this case, the production process is stopped until an expert appears who can resolve the fault that has occurred, causing losses in many areas and thus affecting the economy.

As a result of this situation, the Food Industry Research Institute (IIIA), in conjunction with the Faculty of Computer Engineering and Chemical Engineering of CUJAE, in search of a solution, developed an Expert System for the Control of Chemical Processes ( SECPROIT) and a Knowledge Base Generator.

The SECPROIT expert system aims to evaluate the decisions made by operators in different critical situations and provides managers with their results and how their workers improve in each process. At this time, the system has certain limitations that prevent it from working properly, so it is necessary to update it. Once this objective has been achieved, the problems of the Food Industry, raised at the beginning of this summary, can be resolved.

\vfill

\begin{description}
	\item[Keywords:]{Food Industry, production process, operator, Food Industry Research Institute (IIIA), critical situation, Expert System for the Control of Chemical Processes (SECPROIT), Knowledge Base Generator, expert system.}
\end{description}