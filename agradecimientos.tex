\chapter*{Agradecimientos}
Quiero agradecer a mi familia por todo el apoyo que recibí a lo largo de mi carrera. A mis padres por siempre estar a mi lado. A mi hermano por acercarse a preguntar y darme ánimos. A mis abuelos, por consentirme en todo lo que pueden y por llamar todos los días para preguntar cómo estoy, cómo me siento, si dormí, si comí, cuánto me falta; por ser los mejores abuelos, los amo. A los primos que llamaban para preguntar cómo estoy y cómo me va. A mi tía, que me ayudó siempre que lo necesité. A Alejandro López Rodríguez por estar a mi lado, por darme ánimos y por creer en mi hasta cuando ni yo creía, y a sus padres, por todo el apoyo, la preocupación y la ayuda que me han brindado. Gracias a todos, por ser una familia tan unida y, aunque no somos perfectos, gracias por nunca faltar.

También quiero agradecer a mis amigos, sin ellos soy nada.

Agradecer a Aramays Aimet Morales Durán, que cuando vea su segundo nombre se va a enojar. Ella, más que una amiga, es mi hermana. No solo me consuela y me aconseja, ella fue mi guía todos estos años de universidad. Era la que me motivaba, la que me decía cuando había algún pendiente que olvidé, la que hacía que me ``pusiera para las cosas", fue mi estrella en la facultad.

Agradecer también a Oscar García Páez que, aunque no siga en la carrera es una de las mejores cosas que me pasó. Es mi fiel amigo, el más sincero, el que no teme decirme las verdades a la cara, aunque sepa que no me gusta lo que voy a escuchar. Mi historiador, gracias por ser mi hermano mayor.

A Pavel Pérez González, nuestro ``jefazo". Por ser un gran amigo y compañero, por esos consejos tan únicos e interesantes, por siempre ser atento. Es la persona a la que acudir cuando tienes una duda, porque lo que no sabe, lo aprende con tal de ayudar. Gracias.

A mis amigos y compañeros de aula: Adrián, Abel, Néstor, Daniela, Roly, Darío, Thalía, Amanda y algún otro que se me olvide mencionar. Gracias por la ayuda, las risas y los trabajos en equipo, fue la mejor parte de la universidad.

Gracias a las chicas de la tabla y a mis compañeros de los trece: Claudia, Camila (La flaca), Legna, Yoan, Eduardo (El gato), Víctor, Thalía (La loca), Samuel y María. Junto a ellos defendí el amarillo de mi facultad, participando en deportes que nunca creí jugar. Gracias por enseñarme y cuidarme las lesiones.

 A las chicas del olimpo: Diansy, Thalía, Laura y Arianna, y a mis compañeros de la FEU: Mariam y Lázaro Michel (Tachiri). Cada uno se ocupó de mí, me ayudaron, compartieron conmigo eventos y marchas, y me impulsaron a ser mejor persona. Gracias.
 
Gracias a los amigos que conocí en la pandemia: Mariam Yilian, Anabel Achkienasi, Ronal, Daniela (Mi Stich), Luis Osvel y Laura. A mi compañero de apogones Yamir. En general, gracias a todos los amigos que hice en la universidad, cada uno formó parte de este proceso y son un pedazo de esta etapa tan bonita.

Quiero agradecer a mis profesores, por la formación que me dieron. En especial a Wenny por enseñarme que siempre se puede más, a Rosete por su arte de hacernos amar la carrera, a Mayi por ser como una madre para todos, a Sonia y a Nayma.

Por último, un agradecimiento muy especial a mi tutora, por tenerme tanta paciencia. Gracias por entenderme y apoyarme, aún cuando llevase días desaparecida. Gracias por siempre decirme que puedo y hacerme creer que el mundo es mío.

A todos los que formaron parte de este proceso, de todo corazón: \textbf{Gracias}.