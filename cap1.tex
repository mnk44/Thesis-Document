\chapter{Fundamentos Teóricos}
El presente capítulo abarca los principales temas que se abordan a lo largo de la investigación. Se detalla qué es un sistema de capacitación automatizado, su importancia, sus tipos de preguntas y los modelos de calificación que siguen. Además, se plantea el concepto de sistemas expertos, junto a sus características más notables y su importancia. También estarán explicadas algunas de las funcionalidades del Sistema de Entrenamiento SECPROIT, sus componentes y rendimiento.
Al final del capítulo se observan conclusiones parciales a modo de resumen del mismo.

\section{Sistemas de Capacitación Automatizados}
La capacitación laboral es un método aplicado por las empresas para que su personal adquiera nuevos conocimientos profesionales. Por lo general, se produce ante un ascenso o incorporación, aunque no son los únicos motivos. Lo ideal es que se desarrolle de forma continúa, ya que la constante formación del personal deriva en resultados positivos tanto para el grupo de trabajo como para la organización en la que se realiza \cite{blogBizneo}.
Surge en el mundo como respuesta a la necesidad de mejorar permanentemente la calidad y formación de recursos humanos.

Basándose en el concepto anterior, un sistema de capacitación automatizado, también conocido como sistema de entrenamiento, es un método de enseñanza alternativo creado para el adiestramiento de los trabajadores. Se basa, principalmente, en un software que permite el aprendizaje de los usuarios sin necesidad de una supervisión constante. Generalmente, resulta más efectivo que las prácticas de enseñanza presencial, debido a que el estudiante trabaja solo y puede determinar su propia velocidad de aprendizaje, usando una amplia variedad de herramientas y métodos para la transferencia del conocimiento \cite{seguridadMinera}.

A modo de resumen, es un software que brinda una solución de recursos humanos, ayuda en la formación de los trabajadores y aumenta la productividad empresarial.

\subsection{Características de un Sistema de Capacitación}
Un sistema de capacitación, ya sea automatizado o no, ofrece diferentes aplicaciones en función del modelo de negocio que utiliza. Su versatilidad permite adaptarse a las necesidades particulares de cada sector. Sin embargo, según \cite{softDoit}, la mayoría de los sistemas contienen las mismas características:
\begin{itemize}
\item Son capaces de gestionar los distintos cursos impartidos, la asistencia y la inversión en formación de la empresa.
\item Asignan a los empleados que deberán asistir y los profesionales responsables de analizar sus resultados.
\item Detectan las carencias formativas del personal antes de que influyan en el desarrollo del trabajo.
\item Clasifican las distintas actividades formativas en base a su categoría y catálogo.
\item Registran y consultan el progreso del aprendizaje de los empleados en tiempo real.
\end{itemize}

\subsection{Importancia de los Software de Capacitación}
Un sistema de entrenamiento asistido por computadora permite ofrecer el mismo nivel de adiestramiento para cada usuario del sistema, en cuanto a rigor y evaluación. Uno de los problemas de la capacitación de los empleados de manera presencial, es que las sesiones son frecuentemente inconsistentes, y las diferencias en el nivel de habilidad del formador pueden tener un impacto significativo en el éxito del empleado. Al contar con un sistema automatizado, solo se necesita una base de conocimientos para garantizar el mismo nivel de entrenamiento para todos los capacitados \cite{cap2002}.

\subsection{Fases de un Proceso de Evaluación de Conocimiento}
Un proceso de evaluación de conocimiento, debe estar integrado por cinco etapas (Figura \ref{fig:etapasEval}), asegura \cite{garcia1994}. Cada una de ellas, va a marcar un conjunto de acciones, que al final se interpretarán como un buen entrenamiento:

\begin{itemize}
\item \textsl{Recogida de Datos:} es la recopilación sistemática de toda la información a lo largo del proceso completo de enseñanza-aprendizaje. Debe tener concordancia con los objetivos, ser suficiente, representativa, relevante y ponderada, en función del peso otorgado a cada objetivo. En los sistemas en línea estas posibilidades de registrar evidencias son inmensas.
\item \textsl{Puntuación de las Pruebas:} se realiza una vez medidos, de manera cuantitativa o cualitativa, los distintos bloques de información, con las ponderaciones, criterios e indicadores que se hayan establecido.
\item \textsl{Juicio de Valor:} puede hacerse limitándose a criterios de grupo (evaluación normativa), refiriéndose a criterios de superación de objetivos y/o contenidos (evaluación de criterio), o teniendo en cuenta la personalidad, posibilidades y limitaciones del propio sujeto del aprendizaje (evaluación personalizada).
\item \textsl{Toma de Decisiones:} habitualmente denominada calificación, trae consigo una serie de consecuencias personales, administrativas, económicas y laborales. Se basa en la decisión a partir del resultado. La acción resultante influye directamente en el adiestrado. Ejemplos de decisiones son: selección/exclusión, promoción/recuperación/repetición, rebajar a niveles anteriores, ampliación, certificación/reprobación, entre otras.
\item \textsl{Información a los Interesados:} es la etapa final, que ha de llegar a diferentes destinatarios, aunque principalmente y de forma adecuada, a los capacitados. Es la confirmación de que concluye el entrenamiento.
\end{itemize}

\begin{figure}[h]
\centering
 \includegraphics[width=0.5\linewidth]{imagen/FasesEva.jpg}
 \caption{Etapas para garantizar un buen Sistema de Capacitación.}
 \label{fig:etapasEval} 
\end{figure}

\subsection{Tipos de Preguntas en un Sistema de Capacitación Automatizado}
A medida que avanza el tiempo, se generan nuevos métodos de estudio, y con estos, nuevas formas de preguntar y calificar. Sin embargo, a la hora de diseñar un sistema automatizado, no es menos cierto que existen algunas variantes más sencillas y por ende, más utilizadas. Según \cite{referencia12} los tipos de preguntas que mayormente se emplean en un sistema de este tipo son:
\begin{itemize}
\item \textsl{Verdadero o Falso:} contienen una declaración que se debe indicar si es verdadera o no. Permiten responder en poco tiempo, son fáciles, rápidos de calificar y se corrigen de forma automática.
\item \textsl{Opción Múltiple:} se componen de una pregunta (raíz) con
múltiples respuestas posibles. Pueden incluir múltiples opciones válidas, en cuyo caso, podrían darse por superada al marcar cualquiera de ellas o cuando se marquen todas. Se caracterizan por ser fáciles y rápidas de calificar, se
corrigen automáticamente y se pueden utilizar para evaluar los conocimientos en una amplia gama de contenidos.
\item \textsl{Emparejar, Relacionar u Ordenar:} por lo general se emparejan cada una de las opciones del primer bloque con las opciones dadas en el segundo bloque, o se ordenan bloques de modo que quede una secuencia correcta de acuerdo a un patrón previamente establecido. Se suelen usar en aquellos cursos donde la adquisición de conocimientos muy detallados es un objetivo importante. Son preguntas fáciles de diseñar, rápidas de calificar y se corrigen
automáticamente. Estadísticamente, se tarda más en responder que las preguntas anteriores.
\item \textsl{Respuesta Corta:} basta con que se escriban un par de palabras o una frase sencilla. Una alternativa más común a este tipo de preguntas es la de cubrir espacios en blanco con una palabra. Este tipo de preguntas son muy útiles para que se demuestren los conocimientos basados en hechos o palabras claves. La dificultad para calificarlas depende del estilo que se decida emplear.
\end{itemize}

\subsection{¿Cómo verificar la validez de las respuestas?}
Una vez terminado el entrenamiento, mediante el proceso de evaluación, se debe comprobar cuáles de los resultados obtenidos son correctos y cuáles no. Para ello se deben comparar las respuestas del evaluado con una fuente de confianza, que contenga la información verídica de lo que se está tratando. Estas fuentes de confianza se conocen por el nombre: bases de conocimiento.
A partir de ellas, se verifica si los datos en las respuestas del evaluado coinciden con la información real contenida en la base. Este proceso puede realizarse tanto de manera manual, semi-automática o automática \cite{Christmann2022}.

Al tratarse de un sistema de capacitación automatizado, por lo general, el método utilizado para validar las respuestas es el automático. De esta forma se facilita el trabajo para aquellos que deben evaluar a un personal abundante. Según \cite{AltyJL1984}, la manera más efectiva y eficiente de evaluar estos sistemas es a partir del uso de un Sistema Experto.

\subsection{¿Cómo se evalúa una Capacitación?}
La evaluación y la calificación de una prueba, no pueden depender de un solo instrumento o técnica, ya que de esa forma solo se mide un tipo de aprendizaje. Los criterios para calificar que se designen, serán los porcentajes de valor que se establezcan a cada resultado de las actividades realizadas y a su resultado final. Se debe tomar en cuenta tanto la exactitud de la respuesta, como el proceso que se siguió para llegar a la misma, así como la cantidad de intentos necesarios utilizados para hallar la solución correcta.
Una evaluación posee dos objetivos principales: analizar en qué medida se han cumplido los objetivos, para detectar posibles fallas en el proceso y poder superarlas, y proporcionar la reflexión de los que realizaron el entrenamiento en torno a su propio proceso de aprendizaje (metacognición) \cite{RonaldL.Jacobs2012}.

A modo de resumen, para realizar una correcta evaluación se deben tener en cuenta tantas herramientas como parámetros influyan. En este caso en particular: la puntuación de las respuestas y la cantidad de intentos.


\section{Sistemas Expertos}
Los sistemas expertos resuelven problemas que normalmente son solucionados por expertos humanos. Para resolverlos, estos sistemas necesitan acceder a una importante base de conocimiento sobre el dominio, que debe construirse de la manera más eficiente posible.
Utilizan uno o más mecanismos de razonamiento, para aplicar este conocimiento a los problemas que se le proponen. Cuentan con un mecanismo para explicar a los usuarios, que han confiado en ellos, lo que han hecho \cite{AltyJL1984}.

Una forma de contemplar los sistemas expertos es que representan la mayor parte de la Inteligencia Artificial (IA) aplicada. Un sistema experto en IA se define como un programa informático que tiene la capacidad de representar y razonar sobre el conocimiento \cite{Rasheed2021}.