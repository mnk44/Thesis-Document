\chapter[Capítulo 1: Fundamentos teóricos]{Fundamentos teóricos}
\section{Sistemas de capacitación laboral}
La capacitación laboral es un método aplicado por las empresas para que su personal adquiera nuevos conocimientos profesionales. Por lo general, se produce ante un ascenso o incorporación, aunque no son los únicos motivos. Busca perfeccionar al colaborador en su puesto laboral, en función de las necesidades de su empresa. Es un proceso estructurado con metas bien definidas. Surge en el mundo como respuesta a la necesidad de mejorar permanentemente la calidad y formación de recursos humanos. Lo ideal es que se desarrolle de forma continua, ya que la constante formación del personal deriva en resultados positivos tanto para el grupo de trabajo como para la organización en la que se realiza \cite{Denby2010}.

\subsection{Características de un sistema de capacitación}
Un sistema de capacitación puede ofrecer diferentes aplicaciones en función del modelo de negocio que utilice. Su versatilidad permite adaptarse a las necesidades particulares de cada sector. Sin embargo, según \cite{Paez2022}, la mayoría de las capacitaciones contienen las siguientes características:

\begin{itemize}
\item Son capaces de gestionar los distintos cursos impartidos, la asistencia y la inversión en formación de la empresa
\item Asignan a los empleados que deberán asistir y a los profesionales responsables de analizar sus resultados
\item Detectan las carencias formativas del personal antes de que influyan en el desarrollo del trabajo
\item Clasifican las distintas actividades formativas en base a su categoría y catálogo
\item Registran y consultan el progreso del aprendizaje de los empleados en tiempo real
\end{itemize}

\subsection{Importancia de una buena capacitación}
La capacitación laboral juega un papel primordial para el logro de tareas y proyectos, dado que es el proceso mediante el cual los trabajadores adquieren conocimientos, herramientas, habilidades y actitudes para interactuar de forma correcta y segura en el entorno laboral. Entre los principales beneficios que aporta, según \cite{RogelioE.Martinez2002}, se destacan:

\begin{itemize}
\item Calidad y mejora en el resultado de las tareas
\item Reducción en tiempos de trabajo y supervisión
\item Solución de problemas con diferentes visiones
\item Sensibilización ante nuevos retos
\item Desarrollo ético y motivación del personal
\item Seguridad y autoestima en los trabajadores
\item Mayor especialización
\end{itemize}

\subsection{Proceso de evaluación en una capacitación}
La evaluación de una capacitación no puede depender de un solo instrumento o técnica, ya que de esa forma solo se mide un tipo de aprendizaje. Los criterios para calificar que se designen serán mostrados como porcentajes de valor asociados a cada resultado de las actividades realizadas y a su resultado final. Entre los criterios más comunes que se tienen en cuenta están: la exactitud de la respuesta, el proceso que se siguió para llegar a la misma, la cantidad de intentos necesarios utilizados para hallar la solución y, en algunos casos, el tiempo necesitado para responder \cite{Jacobs2012}.

Una evaluación posee dos propósitos fundamentales: analizar en qué medida se han cumplido los objetivos y proporcionar una reflexión de los que realizaron el entrenamiento en torno a su propio proceso de aprendizaje (metacognición). Analizar el cumplimiento de los objetivos permite detectar posibles fallas en el proceso y poder superarlas en un futuro \cite{Aretio2020}.

A modo de resumen, para obtener una correcta evaluación se deben tener en cuenta tantas herramientas como parámetros influyan.