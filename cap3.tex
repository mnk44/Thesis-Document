\chapter*{Capítulo 3 \vspace{0.5cm} \break Validación del nuevo sistema de capacitación}
\setcounter{chapter}{3}
\setcounter{section}{0}
\addcontentsline{toc}{chapter}{Capítulo 3: Validación del nuevo sistema de capacitación}

Partiendo del objetivo principal de esta investigación, rectificar las limitaciones existentes en el sistema SECPROIT, se desarrollaron un conjunto de pruebas y experimentos para demostrar que se enmendaron dichas restricciones. En el siguiente capítulo se presentan los resultados obtenidos en cada una de las pruebas. Este proceso parte de cómo la fase de análisis y diseño se unen para llevar a cabo el sistema propuesto.

\section{Pruebas funcionales}
 Las pruebas de software, o pruebas funcionales, son el proceso de ejecutar un sistema o componente para medir su calidad, con la intención de encontrar errores que aún no se descubren \cite{Buehler2008}.

Para este software, el nivel de prueba que se utiliza es: pruebas de sistema. Consiste en ejecutar el sistema completo, buscando defectos tanto en aspectos generales como en particulares. Se utilizan pruebas basadas en las funcionalidades (pruebas funcionales) y el método que se emplea es el de caja negra (se lleva a cabo sobre la interfaz del software) \cite{Nidhra2012}.

\subsection{Casos de prueba para la configuración entrenamientos}
Como se pudo apreciar en el capítulo 2, en el sistema desarrollado se introdujeron nuevas entidades, entre ellas: la configuración de un entrenamiento. Con esta configuración el jefe de área puede decidir el tiempo que puede tomar completar el entrenamiento de una etapa, el tipo de preguntas que pueden aparecer, la cantidad de intentos totales permitidos por cada etapa y la cantidad de intentos aprobados necesarios para superar cada una.

Gracias a esta nueva configuración, se lograron resolver algunas de las limitaciones que se mencionaban en el capítulo 1: 
\begin{itemize}
\item Existe más de un modelo de pregunta, solucionando el problema de escasez en el método de aprendizaje
\item Las etapas no aparecen de forma continúa, porque para poder avanzar a la siguiente se debe aprobar la etapa actual un número determinado de veces
\item Solo aparece una etapa a la vez
\item Para cada etapa existe más de un entrenamiento
\end{itemize}

Sin embargo, es necesario probar el correcto funcionamiento de este nueva entidad. Con ese fin, se diseñaron tres casos de prueba  (\textsl{Tabla \ref{cas1}}, \textsl{Tabla \ref{cas2}}, \textsl{Tabla \ref{cas3}}), cada una con sus propios experimentos. 

\subsubsection{Casos de prueba: Insertar configuración de entrenamiento}
El proceso de crear una nueva configuración de entrenamiento se realiza al insertar un nuevo proceso (todo ocurre en la misma interfaz de usuario). Para este caso de prueba se realizaron tres experimentos:

\begin{table}[H]
\begin{center}
\begin{tabular}{ | p{4cm} | p{3.8cm} | p{3.1cm} | p{3.2cm} |}
\hline
\centering\textbf{Prueba} & \textbf{Descripción} & \textbf{Resultado \break esperado} & \textbf{Resultado \break obtenido} \\
\hline
Insertar configuración con campos en blanco & Una configuración de entrenamiento no puede contener información vacía & Los campos están llenos por defecto y la interfaz no debe permitir que se vacíen los campos & La interfaz no permitió que los campos fueran vaciados (Anexo \ref{fig:cb}) \\
\hline
Insertar configuración con campos incorrectos & La única forma de introducir información incorrecta es insertando un número de intentos verídicos mayor que el número de intentos total & Debe aparecer un mensaje de error & Aparece un mensaje de error (Anexo \ref{fig:ci}) \\
\hline
Insertar configuración con campos correctos & Introducir campos correctos & Se debe introducir la configuración  & Se introdujo la nueva configuración (Anexo \ref{fig:cc}) \\
\hline
\end{tabular}
\caption{Casos de prueba: Insertar configuración de entrenamiento}
\label{cas1}
\end{center}
\end{table}

\subsubsection{Casos de prueba: Modificar configuración de entrenamiento}
El proceso de modificar la configuración de un entrenamiento se realiza al modificar un proceso (todo ocurre en la misma interfaz). Para este caso de prueba se realizaron tres experimentos:

\begin{table}[H]
\begin{center}
\begin{tabular}{ | p{4cm} | p{4cm} | p{3.1cm} | p{3cm} |}
\hline
\centering\textbf{Prueba} & \textbf{Descripción} & \textbf{Resultado \break esperado} & \textbf{Resultado \break obtenido} \\
\hline
Modificar configuración con campos en blanco & Una configuración de entrenamiento no puede contener información vacía & Los campos están llenos por defecto y la interfaz no debe permitir que se vacíen los campos & La interfaz no permitió que los campos fueran vaciados \\
\hline
Modificar configuración con campos incorrectos & La única forma de introducir información incorrecta es insertando un número de intentos verídicos mayor que el número de intentos total & Debe aparecer un mensaje de error & Aparece un mensaje de error \\
\hline
Modificar configuración con campos correctos & Introducir campos correctos & Se debe modificar la configuración  & Se modificó la configuración \\
\hline
\end{tabular}
\caption{Casos de prueba: Modificar configuración de entrenamiento}
\label{cas2}
\end{center}
\end{table}

\subsubsection{Casos de prueba: Eliminar configuración de entrenamiento}
Como medida de seguridad en el nuevo sistema de capacitación, el proceso de eliminar la configuración de un entrenamiento no existe de manera visual. Para eliminarla, debe eliminarse el proceso al que esta pertenece. Por lo tanto, el único experimento que se puede realizar en este caso es: eliminar un proceso.

\begin{table}[H]
\begin{center}
\begin{tabular}{ | p{2.7cm} | p{5.2cm} | p{3.1cm} | p{3.2cm} |}
\hline
\centering\textbf{Prueba} & \textbf{Descripción} & \textbf{Resultado \break esperado} & \textbf{Resultado \break obtenido} \\
\hline
Eliminar proceso & Se selecciona de una tabla el proceso que se desea eliminar & Se debe eliminar el proceso & Se eliminó el proceso (Anexo \ref{fig:ce}) \\
\hline
\end{tabular}
\caption{Casos de prueba: Eliminar configuración de entrenamiento}
\label{cas3}
\end{center}
\end{table}

\subsection{Casos de prueba para el entrenamiento en la etapa de las causas}
En el sistema SECPROIT, la etapa de las causas no se evalúa correctamente. Si la variable presenta más de una causa, el sistema evalúa la primera pero no logra evaluar las demás. En cambio, si del grupo de variables solo una se encuentra fuera de rango, la etapa de las causas no se evalúa en absoluto.

Sin embargo, el nuevo sistema desarrollado no presenta esta dificultad. En el software, para cada tipo de pregunta en la etapa de las causas, se evalúan un número distinto de variables (todas fuera de rango). Por ejemplo, en las preguntas de completar los espacios en blanco se evalúan cinco variables distintas (todas fuera de rango) y de cada una se preguntan sus causas (Anexo \ref{fig:pregcaus}}), mientras que en las preguntas de enlazar, solo se pregunta por una variable, pero esta posee múltiples causas.

Para probar el correcto funcionamiento del entrenamiento en la etapa de las causas, se evaluaron cuatro entrenamientos de esta etapa (\textsl{Tabla \ref{cas:causa}}), uno por cada pregunta (verdadero o falso, completar los espacios en blanco, selección múltiple y enlazar). Como las preguntas que se generan en los entrenamientos son aleatorias, para estos casos de prueba el valor de datos de entrada no es relevante. Para afirmar que las respuestas son correctas, se establece una comparación entre las respuestas dadas por el sistema y las respuestas que brinda el motor de reglas (\textsl{JDrools}).

\begin{table}[H]
\begin{center}
\begin{tabular}{ | p{3cm} | p{5.9cm} | p{2.6cm} | p{2.6cm} |}
\hline
\centering\textbf{Prueba} & \textbf{Descripción} & \textbf{Resultado del JDrools} & \textbf{Resultado del sistema} \\
\hline
Verdadero o \break falso & Se presentan 5 variables con sus causas y se debe indicar si la afirmación es verdadera o falsa & Se acertaron 2 afirmaciones &  Se acertaron 2 afirmaciones \\
\hline
Completar los espacios en blanco & Se presentan 5 variables y de 1 a 5 causas, y se debe rellenar el espacio en blanco de manera que quede una expresión verdadera & Se acertaron 3 expresiones &  Se acertaron 3 expresiones \\
\hline
Enlazar & Se presenta 1 variable y 5 causas, y se deben enlazar las que correspondan a esa variable & Se acertaron 4 &  Se acertaron 4 \\
\hline
Selección múltiple & Se presenta 1 variable y 5 causas, y se deben seleccionar las que correspondan a esa variable & Se acertó 1 selección &  Se acertó 1 selección \\
\hline
\end{tabular}
\caption{Casos de prueba para el entrenamiento en la etapa de las causas}
\label{cas:causa}
\end{center}
\end{table}

\subsection{Casos de prueba para el entrenamiento en la etapa de las recomendaciones}
En el sistema SECPROIT, la etapa de las recomendaciones no se evalúa. Al concluir la evaluación de la etapa de las causas, el sistema califica la etapa de las recomendaciones de excelente y se acaba el entrenamiento (Anexo \ref{fig:errorRec}).

Sin embargo, en el nuevo sistema desarrollado si se evalúan las recomendaciones. Para cada tipo de pregunta se evalúan un número distinto de recomendaciones. Por ejemplo, en las preguntas de verdadero o falso se evalúan cinco recomendaciones, mientras que en las preguntas de enlazar, se evalúan de una a cinco.

Para probar el correcto funcionamiento del entrenamiento en la etapa de las recomendaciones, se evaluaron cuatro entrenamientos de esta etapa (\textsl{Tabla \ref{cas:reco}}), uno por cada pregunta. Como las preguntas que se generan en los entrenamientos son aleatorias, para estos casos de prueba el valor de datos de entrada no es relevante. Para afirmar que las respuestas son correctas, se establece una comparación entre las respuestas dadas por el sistema y las respuestas que brinda el motor de reglas (\textsl{JDrools}).

\begin{table}[H]
\begin{center}
\begin{tabular}{ | p{2.7cm} | p{6.2cm} | p{2.6cm} | p{2.6cm} |}
\hline
\centering\textbf{Prueba} & \textbf{Descripción} & \textbf{Resultado del JDrools} & \textbf{Resultado del sistema} \\
\hline
Verdadero o \break falso & Se presentan 5 causas con sus recomendaciones y se debe indicar si la afirmación es verdadera o falsa & Se acertó 1 afirmación &  Se acertó 1 afirmación \\
\hline
Completar los espacios en blanco & Se presentan 5 causas y de 1 a 5 recomendaciones, y se debe rellenar el espacio en blanco de manera que quede una expresión verdadera & Se acertaron 3 expresiones &  Se acertaron 3 expresiones \\
\hline
Enlazar & Se presenta 1 causa y 5 recomendaciones, y se deben enlazar las que correspondan a esa causa & Se acertaron 2 &  Se acertaron 2 \\
\hline
Selección múltiple & Se presentan 1 causa y 5 recomendaciones, y se deben seleccionar las que correspondan a esa causa & Se acertaron 2 selecciones &  Se acertaron 2 selecciones \\
\hline
\end{tabular}
\caption{Casos de prueba para el entrenamiento en la etapa de las recomendaciones}
\label{cas:reco}
\end{center}
\end{table}

\section{}
