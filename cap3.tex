\chapter*{Capítulo 3 \vspace{0.5cm} \break Validación del nuevo sistema de capacitación}
\setcounter{chapter}{3}
\setcounter{section}{0}
\addcontentsline{toc}{chapter}{Capítulo 3: Validación del nuevo sistema de capacitación}

Partiendo del objetivo principal de la investigación (rectificar las limitaciones existentes en el sistema SECPROIT), se desarrollaron un conjunto de pruebas y experimentos para demostrar que se enmendaron dichas restricciones. En el siguiente capítulo se presentan los resultados obtenidos en cada una de las pruebas. Este proceso parte de cómo la fase de análisis y diseño se unen para llevar a cabo el sistema propuesto.

\section{Pruebas funcionales}
 Las pruebas de software, o pruebas funcionales, son el proceso de ejecutar un sistema o componente para medir su calidad, con la intención de encontrar errores que aún no se descubren. %cita1

Para este software, el nivel de prueba que se utiliza es: pruebas de sistema. Consiste en ejecutar el sistema completo, buscando defectos tanto en aspectos generales como en particulares. Se utilizan pruebas basadas en las funcionalidades (pruebas funcionales) y el método que se emplea es el de caja negra (se lleva a cabo sobre la interfaz del software). %cita2

\subsection{Pruebas funcionales}
