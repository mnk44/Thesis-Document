\chapter*{Introducción}
En la actualidad, hablar de tecnología es pensar en herramientas que hace diez años parecían futuristas y muy difíciles de conseguir, sin embargo, hoy es posible lograr cambios significativos en empresas, empleos y hogares, gracias a los avances tecnológicos \cite{Casamitjana2021}. Tanto así que es cada vez más común escuchar hablar sobre la revolución tecnológica, que crece de manera exponencial y pretende transformar por completo el sector industrial. Estos desarrollos van desde la mecanización de tareas hasta los procesos industriales autónomos, en los que no se necesita la intervención humana para realizar el trabajo manual \cite{Rani2022}.

Para mediados del año 2019, solo un 31\% de las industrias en América Latina y el Caribe poseían procesos automatizados o avances digitales. Esta cifra cambió bruscamente al año siguiente, debido a que la pandemia de la COVID-19 dejó una huella devastadora en la economía. Las empresas que presentaban un avance digital pudieron resistir mejor a la crisis, en términos de impacto sobre las ventas, beneficios y trabajadores despedidos. A partir de ese momento, muchas instituciones tomaron conciencia y comenzaron a incluir avances tecnológicos en sus procesos productivos. Hoy en día, el 86\% de las industrias latinoamericanas cuentan con un desarrollo industrial automatizado y se espera que este cambio se mantenga o aumente en el futuro \cite{Yong2021}.

Cuando se habla de las industrias de un país, no se pueden dejar de mencionar las industrias alimentarias. Estas, son el sector de la economía que se ocupa de todos aquellos procesos relacionados con la alimentación de las personas. Están encaminadas a satisfacer la necesidad más básica de la población, sin la cual no podríamos sobrevivir: la nutrición. Mundialmente, poseen como características la gran variedad de materias primas que emplean, el número elevado de procesos que manejan y su continuo crecimiento, relacionado al ritmo con el que crece la población \cite{ValRoman2016}. 

Cuando se detiene de forma no programada un proceso productivo en una empresa, por lo general trae consigo numerosas pérdidas. En el caso de la industria alimentaria, estas afectaciones pueden verse reflejadas en grandes daños económicos ya que, en la mayoría de los casos, este tipo de industrias trabajan con materias primas que no pueden ser reutilizadas. Pero estas no son las únicas afectaciones que se pueden reflejar. En los procesos productivos de estas industrias, se elaboran distintos tipos de alimentos a partir de un producto determinado, donde cada producto pertenece a una escala de riesgo (A, B o C). Esta escala indica la probabilidad de causar daños en la salud (A para alta, B para media y C para baja). Es decir, si se detiene un proceso productivo en la industria alimentaria no solo traería consigo afectaciones en la economía, sino que también puede verse afectada la salud de los trabajadores o incluso de la población \cite{Salas2018}.

La Industria Alimentaria Cubana se caracteriza, además, por poseer jornadas laborales ininterrumpidas y un personal que cambia frecuentemente. Esta situación dificulta la capacitación de sus trabajadores, por lo que no todos los que laboran poseen el mismo nivel de conocimiento. Como consecuencia, al ocurrir un error en un proceso productivo, no siempre se encuentra el experto capaz de solucionar la falla, dejando como única alternativa: detener el proceso \cite{Lemus2018}.

A partir de esta problemática, el Instituto de Investigación de la Industria Alimentaria (IIIA), junto con las facultades de Ingeniería Química e Ingeniería Informática de la Universidad Tecnológica de La Habana (CUJAE), desarrolló dos sistemas de software: un Generador de Bases de Conocimiento y  un Sistema Experto para el Control de Procesos Químicos (SECPROIT). El primero genera bases de conocimiento que contienen toda la información referente a los procesos productivos que pueden ocurrir en una fábrica, estructurada en: variables, causas por las que puedan estar en peligro y recomendaciones (pasos a seguir en caso de accidentes)  \cite{Riveron2017}. En el segundo, se utilizan las bases creadas por el generador para producir entrenamientos y capacitar a los operarios de la fábrica \cite{ElenaAcostaGil2018}. Con estos sistemas, los trabajadores se entrenarían sobre los procesos que maniobran y se reducirían los riesgos de errores por desconocimiento.

Actualmente el sistema SECPROIT posee ciertas limitaciones que impiden que sea implementado en las industrias:
\begin{itemize}
\item Cada vez que se desea utilizar la información obtenida por el generador es extraída directamente desde sus ficheros, lo que genera una demora extra en el sistema
\item Solo existe un modo de pregunta, por lo que resulta redundante el método de evaluación
\item El tiempo utilizado en responder el entrenamiento no influye en la nota final del mismo
\item La etapa de las recomendaciones no se evalúa correctamente, ni muestra los puntos recibidos
\item Cada etapa se evalúa de manera continúa sin oportunidad de una pausa.
\item Si se suspende una etapa el entrenamiento continúa hasta el final, por lo que aparecen etapas innecesariamente
\item Por cada proceso existe un único entrenamiento.
\item Presenta una interfaz gráfica poco vistosas y de colores muy oscuros que generan desagrado en los usuarios
\end{itemize}

La situación problemática presente en la Industria Alimentaria Cubana sigue siendo la misma, puesto que no se ha podido implementar un sistema de entrenamiento estable que solucione el problema. Partiendo de este principio, una solución viable sería desarrollar una actualización funcional del SECPROIT, que pueda resolver el problema de investigación inicial: 
¿Cómo lograr una capacitación total de los operarios, ante la toma de decisiones en una situación crítica?

Por lo tanto, el objetivo general de esta investigación es: 
\textsl{Rectificar las limitaciones existentes en el sistema SECPROIT para capacitar a los operarios ante los procesos productivos de la fábrica.}

A partir de estas restricciones se intuyen un número de cambios que son necesarios llevar a cabo. Estas modificaciones deben realizarse tanto en la lógica, como en la base de datos, en las interfaces de usuario y en la estructura del sistema. Es decir, son requeridas algunas transformaciones que abarcan todas las capas del SECPROIT.  Según \cite{Plecka2013}, modificar un código ajeno ya escrito, para adaptarlo a una nueva estructura y a una nueva base de datos, resulta más complejo que elaborar uno código de cero. Por lo tanto, con el fin de rectificar las limitaciones actuales y generar una actualización del SECPROIT, resulta más factible comenzar una nueva programación, es decir, un nuevo software, pero sin dejar de lado el modelo del sistema ya existente.

Partiendo de esta teoría, se definen los siguientes objetivos específicos y sus tareas correspondientes:
\begin{enumerate}
\item Asimilar los fundamentos teóricos y analizar nuevos aspectos que puedan ser incorporados
\begin{itemize}
\item Extraer los requisitos funcionales y no funcionales que debe cumplir el nuevo sistema
\item Investigar los diferentes tipos de preguntas existentes en un
sistema de entrenamiento y seleccionar los que mejor se ajusten al proceso evaluativo del SECPROIT
\end{itemize}

\item Programar el nuevo sistema
\begin{itemize}
\item Diseñar diagramas auxiliares y modelados UML
\item Implementar la base de datos
\item Desarrollar funcionalidades que permitan la administración del sistema
\item Incorporar la configuración de un entrenamiento asociado a un proceso
\item Implementar la generación de preguntas por cada fase del entrenamiento
\item Desarrollar un proceso de evaluación parcial para cada etapa y la evaluación integral del operario en un proceso
\item Diseñar e incluir una nueva interfaz gráfica
\end{itemize}

\item Validar la herramienta desarrollada
\begin{itemize}
\item Diseñar las pruebas funcionales
\item Ejecutar las pruebas y/o experimentos
\item Documentar los resultados obtenidos
\end{itemize}

\item Desplegar la aplicación en la facultad de Ingeniería Química
\begin{itemize}
\item Redactar la documentación del sistema
\item Diseñar un manual de instalación y un manual de usuario
\item Instalar la nueva solución en la facultad de Ingeniería Química
\end{itemize}
\end{enumerate}

Esta investigación posee como objeto de estudio: los sistemas de entrenamiento automatizados, los sistemas expertos y los sistemas de información basados en reglas de producción. De ellos, se centra la atención en el campo de acción que comprende el Sistema Experto para el Control de Procesos Químicos (SECPROIT).

El aporte de este proyecto está dado tanto en el marco teórico como en el práctico. Para el primero, se construye un sistema de entrenamiento capaz de ejercitar sobre cualquier información que se brinde en sus bases de conocimiento. En la práctica, queda un sistema de software para la capacitación de los operarios de la Industria Alimentaria Cubana.

En cuanto a la estructura de este trabajo, está dividido en tres capítulos. En el capítulo 1 se encuentran los conceptos fundamentales y los antecedentes de esta investigación. El capítulo 2 describe el diseño e implementación de la solución de esta problemática, utilizando diferentes artefactos UML, describiendo los requisitos del sistema y mostrando la vista de la arquitectura y los patrones utilizados. Por último, en el capítulo 3 se presenta la documentación de las pruebas que se le realizaron al software.