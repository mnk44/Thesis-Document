\section*{Resumen}
La Industria Alimentaria en Cuba juega un rol muy importante dentro de la economía, por lo que debe evitarse a toda costa un problema en este sector. Actualmente presenta la necesidad de capacitar a todos sus operarios ante los procesos productivos que se desarrollan, pues consta de muchas irregularidades a la hora de trabajarlos. Cuando existe una falla no siempre están presentes los expertos para su posible solución, ya que no todos los técnicos están preparados para tomar decisiones correctas en determinadas circunstancias. En ese caso, se detiene el proceso productivo hasta que aparezca el experto que pueda resolver la falla ocurrida, provocando pérdidas en numerosas áreas y, afectando así, la economía.

A raíz de esta situación, el Instituto de Investigaciones de la Industria Alimentaria (IIIA), en conjunto con la Facultad de Ingeniería Informática e Ingeniería Química de la CUJAE, en busca de una solución, desarrolló un Sistema Experto para el Control de Procesos Químicos (SECPROIT) y un Generador de Bases de Conocimientos.

El sistema experto SECPROIT tiene como objetivo evaluar las decisiones que toman los operarios ante diferentes situaciones críticas y brinda a los jefe los resultados de las mismas y cómo se van superando sus trabajadores ante cada proceso. En estos momentos, el sistema presenta ciertas limitaciones que impiden su correcto funcionamiento, por lo que es necesario realizar una actualización del mismo. Una vez logrado este objetivo se podrán resolver los problemas de la Industria Alimentaria, planteados al inicio de este resumen.

\vfill

\begin{description}
	\item[Palabras claves:]{Industria Alimentaria, proceso productivo, operario, Instituto de Investigaciones de la Industria Alimentaria (IIIA), situación crítica, Sistema Experto para el Control de Procesos Químicos (SECPROIT), Generador de Bases de Conocimientos, sistema experto.}
\end{description}


