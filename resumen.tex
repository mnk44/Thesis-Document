\section*{Resumen}
La Industria Alimentaria en Cuba juega un rol muy importante dentro de la economía, por lo que debe evitarse a toda costa un problema en este sector. Actualmente presenta la necesidad de capacitar a todos los operarios de sus plantas, ante los procesos productivos que en ellas se desarrollan, pues existen muchas irregularidades a la hora de operar estos procesos. Cuando existe una falla, no siempre están presentes los expertos para su posible solución, ya que no todos los técnicos están preparados para tomar decisiones correctas en determinadas circunstancias. Es por esto, que al ocurrir una falla, se detiene el proceso productivo hasta que aparezca el experto, provocando pérdidas en numerosas áreas y afectando la economía.
A raíz de esta situación el Instituto de Investigaciones de la Industria Alimentaria (IIIA), en conjunto con la facultad de Ingeniería Informática e Ingeniería Química de la CUJAE, desarrolló un Sistema Experto para el Control de Procesos Químicos (SECPROIT) y un Generador de Bases de Conocimientos.

El SECPROIT tiene como objetivo evaluar las decisiones que toman los operarios ante diferentes situaciones críticas y brindar al jefe de área los resultados de las evaluaciones y como se van superando los mismos ante cada proceso. En estos momentos, el sistema presenta ciertas limitaciones que impiden su correcto funcionamiento, por lo que es necesario realizar una actualización del mismo.


\begin{description}
	\item[Palabras claves:]{Industria Alimentaria, proceso productivo, operario, Instituto de Investigaciones de la Industria Alimentaria (IIIA), Sistema Experto para el Control de Procesos Químicos (SECPROIT), Generador de Bases de Conocimientos.}
\end{description}


