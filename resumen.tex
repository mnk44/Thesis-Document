\section*{Resumen}
La capacitación laboral es un método aplicado por las empresas para que sus trabajadores adquiera nuevos conocimientos profesionales. Con el uso de una capacitación automatizada se logran ahorrar recursos de gran valor, como el tiempo y el capital humano.
En lugares donde se trabaja de forma ininterrumpida, contar con un sistema capacitador resulta una ventaja notable para la producción.

La Industria Alimentaria de un país juega un rol sumamente importante dentro de la economía del mismo. En Cuba, esta industria se caracteriza por trabajar de forma constante, con un personal que cambia frecuentemente. Debido a estas características, resulta casi imposible lograr una completa capacitación de los empleados, lo que provoca que no todos posean el mismo nivel de conocimiento y por ende, no todos puedan responder ante situaciones determinadas.

En el año 2019 se creó un software con el objetivo de resolver esta problemática. El Sistema Experto para el Control de Procesos Químicos (SECPROIT) busca capacitar a los operarios ante los procesos productivos de su fábrica, mediante la aplicación de conjunto de entrenamientos. Actualmente el sistema presenta un grupo de limitaciones funcionales que impiden su empleo.

Esta investigación se basa en resolver las limitaciones existentes en el sistema SECPROIT.

%%%%%%%%%%%%%%%%%%%%%%%%%%%%%%

\vfill

\begin{description}
	\item[Palabras claves:]{Industria Alimentaria, capacitación laboral, sistemas de capacitación automatizados, sistemas de entrenamiento, sistemas expertos, Sistema Experto para el Control de Procesos Químicos (SECPROIT), Generador de Bases de Conocimiento}
\end{description}


