\section*{Resumen}
La Industria Alimentaria en Cuba juega un rol muy importante dentro de la economía del país. Cuando existe una falla en su proceso productivo, no siempre están presentes los expertos para su posible solución, ya que no todos los técnicos están preparados para tomar decisiones correctas en determinadas circunstancias. En ese caso, se detiene la producción hasta que aparezca el trabajador que pueda resolver la falla ocurrida, provocando pérdidas en numerosas áreas y sectores. Es por ello que, esta industria, presenta la necesidad de capacitar a todos sus operarios ante los procesos que en ella se desarrollan.

Para brindarle una solución a esta problemática se desarrollaron, por el Instituto de Investigaciones de la Industria Alimentaria (IIIA) y la Facultad de Ingeniería Informática e Ingeniería Química de la CUJAE, un Sistema Experto para el Control de Procesos Químicos (SECPROIT) y un Generador de Bases de Conocimientos.

Este sistema experto SECPROIT, tiene como objetivo evaluar las decisiones que toman los operarios ante diferentes situaciones críticas y brinda a los jefe los resultados de las mismas y cómo se van superando sus trabajadores ante cada proceso. Actualmente, dicho sistema, presenta ciertas limitaciones que impiden su correcto funcionamiento, por lo que es necesario realizar una actualización del mismo.

Una vez logrado este objetivo se podrán resolver los problemas de la Industria Alimentaria, planteados al inicio de este resumen. A esta investigación se le atribuye un valor práctico, ya que este sistema (artefacto de salida) será implementado en las Industrias Azucareras Cubanas.

\vfill

\begin{description}
	\item[Palabras claves:]{Industria Alimentaria, proceso productivo, operario, Instituto de Investigaciones de la Industria Alimentaria (IIIA), situación crítica, Sistema Experto para el Control de Procesos Químicos (SECPROIT), Generador de Bases de Conocimientos, sistema experto, artefacto de salida.}
\end{description}


